% \iffalse      THIS IS A META-COMMENT
%<*dtx>
\ProvidesFile
%========================================================================
                       {NATBIB.DTX}
%========================================================================
%</dtx>
% This is a LaTeX package to modify \cite and \thebibliography for author-year 
%    systems of bibliographic citation; will also work with 
%    numerical systems, allowing simplified style changes for them too.
% Installation:
%    LaTeX this file: creates docstrip installation file natbib.ins
%                         AND the (LaTeX2e) documentation
%    (La)TeX natbib.ins: creates package file natbib.sty 
%                        and optionally the LaTeX 2.09 style natbib209.sty
%                        and doc driver natbib.drv
%    (natbib.ins may be edited as needed)
% Docstrip options available:
%        package - to produce a (LaTeX2e) package .sty file 
%        driver  - to produce a driver file to print the documentation
%        209     - (with package) for style file that runs under LaTeX 2.09 
%        subpack - (with package) for coding included in other packages
%        all     - (with package) to include all author-year systems
%                   else individually with:
%        apalike, newapa, chicago, harvard, authordate, astron
%        agu     - (with package,subpack) for inclusion in aguplus package
%        egs- (with package,subpack) for inclusion in egs package
%        nopreonly - allows \citestyle and \bibpunct to be used anywhere
%--------------------------------------------------------------------------
%<*!subpack>
%<package&209>\def\ProvidesPackage#1#2]
%<package&209>  {\typeout{Style option `#1'#2]}}
%
%  *** Identify the package file:-
%<package&!209>\NeedsTeXFormat{LaTeX2e}[1994/06/01]
%<package>\ProvidesPackage{natbib}
%</!subpack>
%
%  *** Provide command to dislay module version
%<package&subpack>\def\ModuleVersion#1[#2]{}
%<package&subpack>    \ModuleVersion{natbib}
%
%  *** Identify the driver file:-
%<driver>\NeedsTeXFormat{LaTeX2e}
%<driver>\ProvidesFile{natbib.drv}
%
%  *** The DATE, VERSION, and other INFO
%\fi
%\ProvidesFile{natbib}
        [1997/02/05 6.5 (PWD)]
%\iffalse
%<*package>
%<*!subpack>
%%-------------------------------------------------------------------
%% NOTICE:
%% This file may be used for non-profit purposes.
%% It may not be distributed in exchange for money, 
%%   other than distribution costs.
%%   
%% The author provides it `as is' and does not guarantee it in any way.
%% 
%</!subpack>
%</package>
%% Natbib coding copyright (C) 1994--1997 Patrick W. Daly
%<*package>
%<*!subpack>
%% Max-Planck-Institut f\"ur Aeronomie
%% Postfach 20
%% D-37189 Katlenburg-Lindau
%% Germany
%% 
%% E-mail:
%% Internet-- daly@linmpi.dnet.gwdg.de   
%%-----------------------------------------------------------
%</!subpack>
%</package>
%                    END META-COMMENT \fi
% \changes{4.0}{1993 Aug 19}{First documented release}
% \changes{4.1}{1993 Oct 4}{Simplification of \cs{@citeapalk}}
% \changes{4.1a}{1993 Oct 14}{Add \texttt{rev} option for reversed comments 
%                             in \cs{cite}}
% \changes{4.1b}{1993 Oct 18}{Add \cs{bibfont} to list definition =\cs{relax}}
% \changes{4.2}{1993 Oct 22}{Add coding for AGU, NLINPROC}
% \changes{4.2}{1993 Nov 20}{Add more coding for AGU}
% \changes{4.3a}{1994 Feb 24}{First additions for \LaTeXe}
% \changes{5.0}{1994 May 18}{Revised for \LaTeXe{} and 2.09}
% \changes{5.0}{1994 May 18}{Remove obsolete JGR, GRL coding}
% \changes{5.0}{1994 May 18}{Add \cs{citeauthor}, \cs{citeyear}}
% \changes{5.0}{1994 May 18}{Two optional texts for \cs{cite} so \texttt{rev}
%                            option obsolete}
% \changes{5.0}{1994 May 18}{\LaTeXe\ options to select punctuation}
% \changes{5.1}{1994 Jun 22}{Conform to first official release of \LaTeXe}
% \changes{5.1}{1994 Jun 22}{Separate \LaTeX\ and 2.09 files}
% \changes{5.1}{1994 Jun 22}{Put doc driver first}
% \changes{5.2}{1994 Aug 25}{Fix up 2.09 style to run in compatibility mode}
% \changes{5.2}{1994 Aug 25}{\cs{citeauthor}, \cs{citeyear} make BibTeX 
%                             entry in aux file}
% \changes{5.2}{1994 Aug 25}{\cs{@citex} defined as in \LaTeXe}
% \changes{5.2}{1994 Aug 25}{Local config file \texttt{natbib.cfg} read in}
% \changes{5.3}{1994 Sep 13}{Add \cs{citefullauthor}, options \texttt{angle}, 
%                             \texttt{curly}}
% \changes{5.3}{1994 Sep 19}{Add star version of \cs{cite} for full authors}
% \changes{5.3}{1994 Sep 26}{Fix accents in citations with proper definition
%                             of \cs{protect}}
% \changes{5.4}{1994 Nov 24}{Add space in \cs{@citex} for text cites}
% \changes{5.4}{1994 Nov 24}{Replace \cs{if@tempswa} by \cs{ifNAT@swa}}
% \changes{5.4}{1994 Nov 24}{Add superscript citation type to \cs{bibpunct}}
% \changes{5.4}{1994 Nov 24}{Add \cs{@citesuper}, fix up bugs in superscripts}
% \changes{5.4}{1994 Nov 24}{Define \cs{@citexnum} as in \LaTeXe}
% \changes{5.4}{1995 Feb 03}{Add \cs{citestyle} same as \cs{bibstyle}}
% \changes{5.4}{1995 Feb 08}{For repeated years and authors, print just letter}
% \changes{5.5}{1995 Mar 13}{Add \cs{bibhang} and command space in
%      \cs{@cite}}
% \changes{5.5}{1995 Mar 16}{Add \cs{citealt} for citation with no
%      parentheses}
% \changes{5.5}{1995 Mar 24}{Reorganize internal commands, using \cs{NAT@}
%     prefixes}
% \changes{5.5}{1995 Mar 24}{Punctuation selection commands \cs{bibpunct},
%     \cs{citestyle} are now preamble only, whereas previously they had to
%     come after the preamble}
% \changes{5.5}{1995 May 14}{Change names of punctuation commands to
%     \cs{NAT@...}}
% \changes{6.0}{1995 Sep 4}{Allow numerical styles with author-year
%     \texttt{bst} files}
% \changes{6.0}{1995 Sep 21}{Add automatic indexing of citations}
% \changes{6.0}{1995 Sep 29}{Accommodate \texttt{index} package}
% \changes{6.1}{1995 Nov 22}{Fixed for \LaTeXe\ \texttt{1995/12/01}}
% \changes{6.1}{1995 Dec 4}{Make more robust against changes to internals}
% \changes{6.1a}{1995 Dec 19}{Fix test for changed citations}
% \changes{6.2}{1996 Jan 10}{Replace all \cs{uppercase}}
% \changes{6.2}{1996 Jan 11}{Add \cs{citet}}
% \changes{6.2}{1996 Feb 2}{Fix superscript size}
% \changes{6.2}{1996 Mar 05}{Add length \cs{bibsep} for linespacing between
%               references}
% \changes{6.2}{1996 Apr 15}{Fix clash with \texttt{amsart} and
%              \texttt{amsbook}}
% \changes{6.3}{1996 Jun 10}{Allow \texttt{showkeys} to be loaded first}
% \changes{6.3}{1996 Jun 17}{Fix punctuation for \texttt{plainnat}}
% \changes{6.3}{1996 Jun 17}{Suppress extra labels with numericals}
% \changes{6.4}{1996 Jun 18}{Provide \cs{bibname} and \cs{refname}}
% \changes{6.4}{1996 Jun 27}{Change \texttt{nlinproc} option to \texttt{egs}}
% \changes{6.4}{1996 Sep 1}{Make compatible with \texttt{chapterbib.sty}}
% \changes{6.4}{1996 Sep 12}{Fix spacing for superscripts}
% \changes{6.4}{1996 Sep 12}{Extra letter printed with \cs{citeyear}}
% \changes{6.4}{1996 Sep 13}{Add compression and sorting of numerical citations}
% \changes{6.4}{1996 Oct 2}{Make compatible with \texttt{hyperref.sty}}
% \changes{6.5}{1996 Dec 11}{KOMA script compatibility}
% \changes{6.5}{1997 Jan 10}{For EGS, no blank line between references}
% \changes{6.5}{1997 Jan 30}{Recode notes so they work with \cs{citet} and
%     \cs{citep}; change documentation to stress these commands over \cs{cite}}
% \changes{6.5}{1997 Feb 5}{Fix KOMA script properly}
%
% \CheckSum{1706}
% \CharacterTable
%  {Upper-case    \A\B\C\D\E\F\G\H\I\J\K\L\M\N\O\P\Q\R\S\T\U\V\W\X\Y\Z
%   Lower-case    \a\b\c\d\e\f\g\h\i\j\k\l\m\n\o\p\q\r\s\t\u\v\w\x\y\z
%   Digits        \0\1\2\3\4\5\6\7\8\9
%   Exclamation   \!     Double quote  \"     Hash (number) \#
%   Dollar        \$     Percent       \%     Ampersand     \&
%   Acute accent  \'     Left paren    \(     Right paren   \)
%   Asterisk      \*     Plus          \+     Comma         \,
%   Minus         \-     Point         \.     Solidus       \/
%   Colon         \:     Semicolon     \;     Less than     \<
%   Equals        \=     Greater than  \>     Question mark \?
%   Commercial at \@     Left bracket  \[     Backslash     \\
%   Right bracket \]     Circumflex    \^     Underscore    \_
%   Grave accent  \`     Left brace    \{     Vertical bar  \|
%   Right brace   \}     Tilde         \~}
%
% \iffalse
%<*install>
%^^A =============================================
%^^A    Here is the docstrip installation file
%^^A    It is written on first LaTeX run
%^^A =============================================
\begin{filecontents}{natbib.ins}
% Simply TeX or LaTeX this file to extract various files from 
%  the source file `natbib.dtx'
% Comment out the \generateFile lines for any files you do not want.

\def\batchfile{natbib.ins}
\input docstrip

\preamble

Only a few abbreviated comments remain here to describe the usage.

\endpreamble
\postamble
<<<<< End of decommented file <<<<<<
\endpostamble
\keepsilent

\generateFile{natbib.sty}{f}{\from{natbib.dtx}{package,all}}

\preamble

This is the LaTeX 2.09 version of the package.
It will not be maintained forever.
Please upgrade to LaTeX2e so that you may use the proper package.

Only a few abbreviated comments remain here to describe the usage.

\endpreamble

% Activate the next line to get a version for running under LaTeX 2.09
%\generateFile{natbib209.sty}{f}{\from{natbib.dtx}{package,all,209}}

\preamble

This is the driver file to produce the LaTeX documentation from the 
source file \inFileName.
Make changes to it as needed. (Never change the \inFileName file!)
\endpreamble
\postamble

End of documentation driver file.
\endpostamble

% Activate the next line for a documentation driver 
%   that you can configure yourself.
%\generateFile{natbib.drv}{f}{\from{natbib.dtx}{driver}}

\obeyspaces
\Msg{*******************************************}%
\Msg{* For documentation, process natbib.dtx   *}%
\Msg{*    or the driver file      natbib.drv   *}%
\Msg{* (LaTeX2e only!)                         *}%
\Msg{*******************************************}
\end{filecontents}
%</install>
%<*driver>
\documentclass{ltxdoc}
%<driver>%\documentclass[twoside]{ltxdoc}
%<driver>%\documentclass[a4paper]{ltxdoc}
%<driver>%\documentclass[twoside,a4paper]{ltxdoc}
\raggedbottom

 %** To include the detailed explanation of the coding, comment out
 %**   the next line
\OnlyDescription

 %** To produce a command index: add the following line for one run,
 %**   then run  makeindex -s gind.ist natbib
 %**   and reprocess, with or without this line (much faster without)
%<driver>% \EnableCrossrefs\CodelineIndex 

 %** To produce a change history: add the following line for one run,
 %**   then run  makeindex -s gglo.ist -o natbib.gls natbib.glo
 %**   and reprocess, with or without this line (faster without)
%<driver>% \RecordChanges 

\DisableCrossrefs %May stay; zapped by \EnableCrossrefs
\CodelineNumbered %May stay

\begin{document}
   \DocInput{natbib.dtx}
\end{document}
%</driver>
%\fi
%
% \DoNotIndex{\begin,\CodelineIndex,\CodelineNumbered,\def,\DisableCrossrefs}
% \DoNotIndex{\DocInput,\documentclass,\EnableCrossrefs,\end,\GetFileInfo}
% \DoNotIndex{\NeedsTeXFormat,\OnlyDescription,\RecordChanges,\usepackage}
% \DoNotIndex{\ProvidesClass,\ProvidesPackage,\ProvidesFile,\RequirePackage}
% \DoNotIndex{\LoadClass,\PassOptionsToClass,\PassOptionsToPackage}
% \DoNotIndex{\DeclareOption,\CurrentOption,\ProcessOptions,\ExecuteOptions}
% \DoNotIndex{\AtEndOfClass,\AtEndOfPackage,\AtBeginDocument,\AtEndDocument}
% \DoNotIndex{\InputIfFileExists,\IfFileExists,\ClassError,\PackageError}
% \DoNotIndex{\ClassWarning,\PackageWarning,\ClassWarningNoLine}
% \DoNotIndex{\PackageWarningNoLine,\ClassInfo,\PackageInfo,\MessageBreak}
% \DoNotIndex{\space,\protect,\DeclareRobustCommand,\CheckCommand}
% \DoNotIndex{\newcommand,\renewcommand,\providecommand,\newenvironment}
% \DoNotIndex{\renewenvironment,\newif,\newlength,\newcounter,\setlength}
% \DoNotIndex{\setcounter,\if,\ifx,\ifcase,\ifnum,\ifdim,\else,\fi}
% \DoNotIndex{\texttt,\textbf,\textrm,\textsl,\textsc,\reset@font}
% \DoNotIndex{\textup,\textit,\textmd,\textsf,\emph,\futurelet}
% \DoNotIndex{\ttfamily,\rmfamily,\sffamily,\mdseries,\bfseries,\upshape}
% \DoNotIndex{\slshape,\scshape,\itshape,\em,\LaTeX,\LaTeXe}
% \DoNotIndex{\filename,\fileversion,\filedate,\let,\makeindex}
% \DoNotIndex{\@auxout,\@for,\@gobble,\@ifnextchar,\@m,\@mkboth,\@nil}
% \DoNotIndex{\@noitemerr,\@tempa,\@tempswafalse,\@tempswatrue,\@warning}
% \DoNotIndex{\advance,\arabic,\AtBeginDocument,\bf,\bibname,\chapter}
% \DoNotIndex{\citation,\clubpenalty,\CodelineNumbered,\csname}
% \DoNotIndex{\DisableCrossrefs,\do,\edef,\else,\endcsname,\endlist}
% \DoNotIndex{\expandafter,\fi,\gdef,\global,\hbox,\hfill,\hskip,\hspace}
% \DoNotIndex{\if,\if@filesw,\if@tempswa,\ifx,\immediate,\itemindent,\labelsep}
% \DoNotIndex{\labelwidth,\lastskip,\leftmargin,\list,\mbox,\newblock}
% \DoNotIndex{\newpage,\p@enumiv,\parindent,\penalty,\refname}
% \DoNotIndex{\relax,\section,\settowidth,\sfcode,\sloppy,\small,\string}
% \DoNotIndex{\theenumiv,\thepage,\unskip,\uppercase,\usecounter,\vskip}
% \DoNotIndex{\widowpenalty,\write,\xdef,\z@,\catcode,\ifnum,\the}
% \DoNotIndex{\.,\@empty,\@ifundefined,\@latex@warning,\@minus,\@plus,\ }
% \DoNotIndex{\document,\@namedef,\@listi}
% \DoNotIndex{\listparindent,\noexpand,\par,\parsep,\pb,\pbf,\pbfseries}
% \DoNotIndex{\pc,\pd,\pem,\pit,\pitshape,\pmdseries,\prm,\prmfamily,\psc}
% \DoNotIndex{\pscshape,\psf,\psffamily,\psl,\pslshape,\ptt,\pttfamily}
% \DoNotIndex{\pupshape,\@iden,\@firstofone,\@unexpandable@protect}
% \DoNotIndex{\&,\{,\},\bibitem,\bibindent,\if@draft,\typeout}
% \DoNotIndex{\@ifclassloaded,\@ifstar,\@onlypreamble,\@preamblecmds}
% \DoNotIndex{\addtolength,\endinput,\@bsphack,\begingroup,\@wrindex}
% \DoNotIndex{\@listctr,\bibname,\enddocument,\hfil,\ignorespaces,\item}
% \DoNotIndex{\NAT@temp,\refname,\stepcounter,\@ifpackageloaded}
% \DoNotIndex{\@gobbletwo,\index,\itemsep,\markright,\scriptsize}
% \DoNotIndex{\textsuperscript}
%
% \setcounter{IndexColumns}{2}
% \setlength{\IndexMin}{10cm}
% \setcounter{StandardModuleDepth}{1}
%
% \hyphenation{par-en-the-ti-cal}
%
% \GetFileInfo{natbib}
%
% \title{{\bfseries Natural Sciences Citations and References}\\
%         (Author--Year and Numerical Schemes)}
%    
% \author{Patrick W. Daly}
%         
% \date{This paper describes package \texttt{\filename}\\
%       version \fileversion{} from \filedate\\[1ex]
%^^A     \textsl{It is part of the \texttt{preprint} collection of packages}\\
%     \textbf{It is intended for \LaTeXe}
%     \\ (but does include a variant for \LaTeX~2.09)
%  }
% 
% \maketitle
%
% \pagestyle{myheadings}
% \markboth{P. W. Daly}{NATURAL SCIENCES CITATIONS AND REFERENCES}
%
% \newcommand{\btx}{\textsc{Bib}\TeX}
% \newcommand{\thestyle}{\texttt{\filename}}
%
%^^A In order to keep all marginal notes on the one (left) side:
%^^A (otherwise they switch sides disasterously with twoside option)
% \makeatletter \@mparswitchfalse \makeatother 
%
% \begin{abstract}
% Journals in the natural sciences tend to use the author--year style of
% literature citations, in contrast to the numerical style supported by
% \LaTeX{} and \btx. A number of contributed packages exist to accommodate this
% citation mode, but each one is tailored to a limited number of \btx\ style
% (\texttt{.bst} files. The reason is that each one reads in the author--year
% information in different formats for the |\bibitem| commands.
% In contrast, 
% the \thestyle{} package supports not only the various author--year
% bibliography styles, but also those for standard numerical citations. In
% fact, it can also produce numerical citations even with an author--year
% bibliographic style, something that permits easy switching between the
% two citation modes. To this end, replacements for the standard
% \LaTeX\ \texttt{.bst} files are also provided.
% 
% It is possible to define the citation \emph{style} (type of brackets and
% punctuation between citations) and even to associate it with the name of the
% bibliographic style so that it is automatically activated. Citation styles
% can be defined for local \texttt{.bst} files by means of a configuration file
% \thestyle\texttt{.cfg}.
%  
% It is compatible with the packages: \texttt{index}, \texttt{showkeys}, 
% \texttt{chapterbib}, \texttt{hyperref},
% and with the classes \texttt{amsbook} and
% \texttt{amsart}. It can also emulate the sorting and compressing
% functions of the \texttt{cite} package (with which it is otherwise
% incompatible).
%
% The \thestyle\ package therefore acts as a single, flexible interface for 
% most of the available bibliographic styles.
%
% \end{abstract}  
% 
% \section{Introduction}
% The first problem of using author--year literature citations with standard
% \LaTeX{} is that the two forms of citations are not supported. These are:
% \begin{quote}
% textual: \dots\ as shown by Jones et al. (1990) \dots\\
% parenthetical: It has been shown (Jones et al., 1990) that \dots
% \end{quote}
% There is only one |\cite| command to do both jobs.
% 
% A second problem is that the \texttt{thebibliography} environment for
% listing the references insists on including the {\em labels\/} in the
% list. These labels are normally the numbers, needed for referencing. In
% the author--year system, they are superfluous and should be left off.
% Thus, if one were to make up a bibliography with the author--year as
% label, as
% \begin{quote}
% \begin{verbatim}
% \begin{thebibliography}{...}
% \bibitem[Jones et al., 1990]{jon90}
% Jones, P. K., . . .
% \end{thebibliography}
% \end{verbatim}
% \end{quote}
% then |\cite{jon90}| produces the parenthetical citation [Jones et al.,
% 1990], but there is no way to get the textual citation. Furthermore,
% the citation text will also be included in the list of references. 
% 
% The final problem is to find a \btx{} bibliography style that will be
% suitable.
% 
% \section{Previous Solutions}
% \begin{quote}\slshape
% This section may not be of interest to all users. To find out how to use
% \thestyle\ without reading about the historical background, go to
% Section~\ref{sec:usage}.
% \end{quote}
% 
% Although the author--year citation mode is not supported by
% \emph{standard} \LaTeX, there are a number of contributed packages that
% try to solve this problem. The various bibliographic styles
% (\texttt{.bst} files) that exist are usually tailored to be used with a
% particular \LaTeX{} package.
% 
% I have found a large number of \texttt{.bst} files on file servers that may
% act as indicators of the various systems available.
% 
% \subsection{The \texttt{natsci.bst} Style}
% What gave me my first inspiration was Stephen Gildea's \texttt{natsci.bst}
% for use with his \texttt{agujgr.sty} file. This showed me that the problem
% was solvable. However, Gildea's style formats |\bibitem| just as I
% illustrated above: with an optional label consisting of abbreviated
% authors and year. Thus only parenthetical citations can be accommodated.
% The list of references, however, is fixed up in his style files.
% 
% \subsection{The \texttt{apalike.bst} Style}
% Oren Patashnik, the originator of \btx{} and the standard \texttt{.bst}
% files, has also worked on an author--year style, called \texttt{apalike.bst}
% with a corresponding \texttt{apalike.sty} to support it. Again, only the
% parenthetical citation is provided. Except for the fact that his style
% works with version~0.99 of \btx, its functionality is identical to that
% of the \texttt{natsci} files.
% 
% Patashnik does not like author--year citations. He makes this very clear
% in his \btx{} manuals and in the header to \texttt{apalike.bst}.
% Nevertheless, one should respect his work in this area, simply because he
% should be the best expert on matters of \btx. Thus \texttt{apalike.bst}
% could be the basis for other styles.
% 
% The form of the \texttt{thebibliography} entries in this system is
% \begin{quote}
% |\bibitem[Jones et al., 1990]{jon90}...|
% \end{quote}
% the same as I illustrated earlier. This is the most minimal form that can
% be given. I name it the \texttt{apalike} variant, after Patashnik's 
% \texttt{apalike.bst} and \texttt{apalike.sty}. However, there could be many
% independent \texttt{.bst} files that follow this line.
% 
% The bibliography style files belonging to this group include:
% \begin{quote}
% \texttt{apalike}, \texttt{apalike2}, \texttt{cea}, \texttt{cell}, 
% \texttt{jmb}, \texttt{phapalik}, \texttt{phppcf}, \texttt{phrmp}
% \end{quote}
% 
% \subsection{The \texttt{newapa} Style}
% A major improvement has been achieved with \texttt{newapa.bst} and the
% accompanying \texttt{newapa.sty} files by Stephen N. Spencer and Young U.
% Ryu. Under their system, three separate items of information are included
% in the |\bibitem| label, to be used as required. These are: the full
% author list, the abbreviated list, and the year. This is accomplished by
% means of a |\citeauthoryear| command included in the label, as
% \begin{quote}
% |\bibitem[\protect\citeauthoryear{Jones, Barker,|\\
% |  and Williams}{Jones et al.}{1990}]{jon90}...|
% \end{quote}
% Actually, this only illustrates the basic structure of |\citeauthoryear|;
% the \texttt{newapa} files go even further to replace some words and 
% punctuation
% with commands. For example, the word `and' above is really
% |\betweenauthors|, something that must be defined in the \texttt{.sty} file.
% Of course, |\citeauthoryear| is also defined in that file. A
% number of different |\cite| commands are available to print out the
% citation with complete author list, with the short list, with or without
% the date, the textual or parenthetical form. 
% 
% Thus the |\citeauthoryear| entry in |\bibitem| is very flexible,
% permitting the style file to generate every citation form that one might
% want. It is used by a number of other styles, with corresponding
% \texttt{.sty} files. They all appear to have been inspired by 
% \texttt{newapa.bst}, although they lack the extra punctuation commands. 
% 
% Bibliographic style files belonging to the \texttt{newapa} group include
% \begin{quote}
% \texttt{newapa}, \texttt{chicago}, \texttt{chicagoa}, \texttt{jas99}, 
% \texttt{named}
% \end{quote}
% Note: the last of these, \texttt{named.bst}, uses |\citeauthoryear| in a
% slightly different manner, with only two arguments: the short list and
% year.
% 
% \subsection{The Harvard Family}
% The same effect is achieved by a different approach in the Harvard family
% of bibliographic styles. Here a substitute for |\bibitem| is used, as
% \begin{quote}
% |\harvarditem[Jones et al.]{Jones, Baker, and|\\
% |   Williams}{1990}{jon90}...|
% \end{quote}
% The accompanying interface package file is called \texttt{harvard.sty}
% and is written by Peter Williams and Thorsten Schnier. It
% defines |\harvarditem| as well as the citation commands |\cite|, for
% parenthentical, and |\citeasnoun|, for textual citations. The first
% citation uses the long author list, following ones the shorter list, if
% it has been given in the optional argument to |\harvarditem|.
% 
% Bibliography styles belonging to the Harvard family are
% \begin{quote}
% \texttt{agsm}, \texttt{dcu}, \texttt{kluwer}
% \end{quote}
% 
% This package has been updated for \LaTeXe, with many additions to
% add flexibility. The result is a powerful interface that should meet most
% citation needs. (It does not suppress repeated authors, though,
% as \thestyle{} does.)
%
% \subsection{The Astronomy Style}
% Apparently realizing the limitations of his \texttt{apalike} system, Oren
% Patashnik went on to develop a `true' \texttt{apa} bibliographic style,
% making use of the method already employed by an astronomy journal. This
% is actually very similar to the \texttt{newapa} label but with only the
% short list of authors:
% \begin{quote}
% |\bibitem[\protect\astroncite{Jones et al.}{1990}]{jon90}|\\
% |   ...|
% \end{quote}
% It requires the package file \texttt{astron.sty} 
% or any other style that defines |\astroncite| appropriately.
% 
% Bibliographic styles belonging to the astronomy group are
% \begin{quote}
% \texttt{apa}, \texttt{astron}, \texttt{bbs}, \texttt{cbe}, 
% \texttt{humanbio}, \texttt{humannat}, \texttt{jtb}
% \end{quote}
% 
% This is as good as the |\citeauthoryear| command, although not as
% flexible since the full list of authors is missing.
% 
% \subsection{The \texttt{authordate} Style}
% Finally, I have also found some packages making use of a label command
% called |\citename| in the form
% \begin{quote}
% |\bibitem[\protect\citename{Jones et al., }1990]{jon90}|\\
% |    ...|
% \end{quote}
% 
% This is not a good system since the author list and date are not cleanly
% separated as individual arguments, and since the punctuation is included
% in the label text. It is better to keep the punctuation fully removed, as
% part of the definitions in the \texttt{.sty} file, for complete flexibility.
% 
% Bibliographic styles belonging to this group are
% \begin{quote}
% \texttt{authordate1}, \texttt{authordate2}, \texttt{authordate3}, 
% \texttt{authordate4}, \texttt{aaai-named}
% \end{quote}
% with accompanying style file \texttt{authordate1-4.sty}.
% 
% \section{The \thestyle{} System}
% The form of the |\bibitem| entry that I have used for all my
% bibliographic styles is only slightly more complicated than the minimal
% one, but allows a clean separation between authors and date:
% \begin{quote}
% |\bibitem[Jones et al.(1990)]{jon90}...|\\[1ex]
% or alternatively\\[1ex]
% |\bibitem[Jones et al.(1990)Jones, Baker, |\\
% \hspace*{2em}|and Williams]{jon90}...|
% \end{quote}
%
% (One weakness of the \thestyle{} format is that it fails if the author
% list itself contains parentheses! This may be fixed up if the author list
% is grouped in curly braces.)
% 
% I wanted to name the system something like `natural sciences bibliography',
% intending it to be a variant of \texttt{natsci.sty}. Since that name
% was already taken, I resorted to the rather cryptic, and definitely ugly,
% \thestyle.
% 
% The \thestyle\texttt{.sty} package\footnote{Formerly called a \emph{style 
% file} in the older \LaTeX~2.09 terminology.}
% supports not only my own |\bibitem| format, but
% also all the others described here, plus numerical citation modes.
% The additional questions of citation style (type of brackets, commas or
% semi-colons between citations) can be defined once and for all for each
% \texttt{.bst} file and need never be specified explicitly in the source text.
% The use of |\cite| is the same for all citation styles, meaning that the
% additional features that might be available in the `proper' \texttt{.sty}
% file will be missing. (This could be changed later.) The result is a
% single \LaTeX{} package to handle {\em all\/} the bibliographic
% styles in a uniform manner.
% 
% As of version 5.1 (1994 June 22), the source file contains coding for a
% \LaTeXe{} package file (the new standard) as well as that for an older
% \LaTeX~2.09 style option file. The latter is extracted with the
% \texttt{docstrip} option \texttt{209}. 
% 
% \begin{quote}\slshape
% \textbf{New to version 6.0}: (1995 Sep 29) \\
% In previous versions, each \texttt{.bst} file was either numerical or
% author--year only. Applying the wrong mode led to grotesque results, and
% the mode could not always be selected automatically. With version~6.0,
% all the author--year bibliographic style files can also be used for
% \emph{numerical} citations, by simply selecting the mode in one of the
% ways described in Sections~\ref{sec:bibpunct} and \ref{sec:opts}. It is
% not possible to employ author-year citations with pure numerical
% \texttt{.bst} files, and never will be. 
% 
% See Section~\ref{sec:6.0} for more information.
% \end{quote}
% 
% \section{Using this Package}\label{sec:usage}
% 
% In this paper, I distinguish between the citation \emph{mode} (author--year
% or numerical) and citation \emph{style} (the type of punctuation used for
% citations). The citation style is something that is independent of the
% bibliography style and is not programmed in the \texttt{.bst} files.
% 
% \subsection{Basic Citation Commands}
% 
% \DescribeMacro{\citet}
% \DescribeMacro{\citep}
% The \thestyle{} package has two basic citation commands, |\citet| and
% |\citep| for \emph{textual} and \emph{parenthetical} citations, respectively.
% There also exist the starred versions |\citet*| and |\citep*| that print
% the full author list, and not just the abbreviated one.
% All of these may take one or two optional arguments to add some text before
% and after the citation.
% \begin{quote}
% \begin{tabular}{l@{\quad$\Rightarrow$\quad}l}
%   |\citet{jon90}| & Jones et al. (1990)\\
%   |\citet[chap.~2]{jon90}| & Jones et al. (1990, chap.~2)\\[0.5ex]
%   |\citep{jon90}| & (Jones et al., 1990)\\
%   |\citep[chap.~2]{jon90}| & (Jones et al., 1990, chap.~2)\\
%   |\citep[see][]{jon90}| & (see Jones et al., 1990)\\
%   |\citep[see][chap.~2]{jon90}| & (see Jones et al., 1990, chap.~2)\\[0.5ex]
%   |\citet*{jon90}| & Jones, Baker, and Williams (1990)\\
%   |\citep*{jon90}| & (Jones, Baker, and Williams, 1990)
% \end{tabular}
% \end{quote}
% The starred versions can only list the full authors if the \texttt{.bst}
% file supports this feature; otherwise, the abbreviated list is printed.
%
% In standard \LaTeX, the |\cite| command can only take a single optional
% text for a note after the citation; here, a single optional text is a
% post-note, while two are the pre- and post-note. To have only a pre-note, it
% is necessary to provide an empty post-note text, as shown above.
% 
% Multiple citations may be made as usual, by including more than one
% citation key in the |\cite| command argument. \textsl{If adjacent citations
% have the same author designation but different years, then the author
% names are not reprinted.}
% \begin{quote}
% \begin{tabular}{l@{\quad$\Rightarrow$\quad}l}
%   |\citet{jon90,jam91}| & Jones et al. (1990); James et al. (1991)\\
%   |\citep{jon90,jam91}| & (Jones et al., 1990; James et al. 1991)\\
%   |\citep{jon90,jon91}| & (Jones et al., 1990, 1991)\\
%   |\citep{jon90a,jon90b}| & (Jones et al., 1990a,b)
% \end{tabular}
% \end{quote}
%
% These examples are for author--year citation mode. In numerical mode, the 
% results are different.
% \begin{quote}
% \begin{tabular}{l@{\quad$\Rightarrow$\quad}l}
%   |\citet{jon90}| & Jones et al. [21]\\
%   |\citet[chap.~2]{jon90}| & Jones et al. [21, chap.~2]\\[0.5ex]
%   |\citep{jon90}| & [21]\\
%   |\citep[chap.~2]{jon90}| & [21, chap.~2]\\
%   |\citep[see][]{jon90}| & [see 21, 1990]\\
%   |\citep[see][chap.~2]{jon90}| & [see 21, chap.~2]\\[0.5ex]
%   |\citep{jon90a,jon90b}| & [21, 32]
% \end{tabular}
% \end{quote}
% The authors can only be listed if the \texttt{.bst} file supports
% author--year citations. The standard \texttt{.bst} files, such as
% \texttt{plain.bst} are numerical only and transfer no author--year
% information to \LaTeX. In this case, |\citet| prints ``\textbf{(author?)} 
% [21].''
% 
% {\slshape Do not use multiple citations with the |\citet| command in
% numerical mode!}
% 
% \DescribeMacro{\cite}
% In the original versions of \thestyle, the traditional |\cite| command was
% used for both textual and parenthetical citations. The presence of an empty
% optional text in square brackets signaled parenthetical. This syntax has been
% retained for compatibility, but is no longer encouraged.
% 
% This means that |\cite| (without notes) is the same as |\citet| in
% author--year mode, whereas in numerical mode, it is the same as |\citep|.
% The starred version, as well as the one or two optional notes, may also be
% used.
%
% \subsection{Extended Citation Commands}
% 
% \DescribeMacro{\citealt}
% As an alternative form of citation, |\citealt| is the same as |\citet|
% but \emph{without any parentheses}. Multiple references and the
% starred variant also exist.
% \begin{quote}
% \begin{tabular}{l@{\quad$\Rightarrow$\quad}l}
%   |\citealt{jon90}| & Jones et al. 1990\\
%   |\citealt*{jon90}| & Jones, Baker, and Williams 1990\\
%   |\citealt{jon90,jam91}| & Jones et al. 1990; James et al. 1991
% \end{tabular}
% \end{quote}
%
% \DescribeMacro{\citeauthor}
% \DescribeMacro{\citeyear}
% \DescribeMacro{\citefullauthor}
% In author--year schemes, it is sometimes desirable to be able to refer to
% the authors without the year, or vice versa. This is provided with three
% extra commands
% \begin{quote}
% \begin{tabular}{l@{\quad$\Rightarrow$\quad}l}
%   |\citeauthor{jon90}| & Jones et al.\\
%   |\citefullauthor{jon90}| & Jones, Baker, and Williams\\
%   |\citeyear{jon90}|   & 1990
% \end{tabular}
% \end{quote}
% If the full author information is missing, then |\citefullauthor| is
% the same as |\citeauthor|, printing only the abbreviated list.
% This also applies to the starred versions of |\citet| and |\citep|.
% 
% If the author or year information is missing (as is the case with the
% standard \LaTeX{} \texttt{.bst} files), these commands issue a warning.
% 
% As of version~6.0, these commands may also be used with numerical
% citations, provided an author--year \texttt{.bst} file is being employed.
% 
% The native \thestyle{} form of the |\bibitem| entry now 
% also supports the full author list.
%
% \textsl{Multiple citations are \emph{not} allowed with these commands.}
% 
% \subsection{Selecting Citation Punctuation}\label{sec:bibpunct}
%
% \DescribeMacro{\bibpunct}
% The above examples have been printed with the default citation style.
% It is possible to change this, as well as to select numerical or
% author--year mode, by means of the |\bibpunct| command, which takes 
% one optional and 6 mandatory arguments. The mandatory ones are:
% \begin{enumerate}
% \item the opening bracket symbol, default = (
% \item the closing bracket symbol, default = )
% \item the punctuation between multiple citations, default = ;
% \item the letter `n' for numerical style, or `s' for numerical superscript
%       style, any other letter for
%       author--year, default = author--year; note, it is not necessary to
%       specify which author--year interface is being used, for all will be
%       recognized;
% \item the punctuation that comes between the author names and the year
%       (parenthetical case only), default = ,
% \item the punctuation that comes between years when common author lists
%       are suppressed (default = ,); if both authors and years are common,
%       the citation is printed as `1994a,b', but if a space is wanted between
%       the extra letters, then include the space in the argument, as |{,~}|.
% \end{enumerate}
% 
% The optional argument is the character preceding a post-note, default is a
% comma.
%
% The |\bibpunct| command must be issued in the preamble, that is,
% before |\begin{document}|. (\textbf{Note:} in versions 5.4 or earlier, this
% declaration had to be given after the preamble!)
% 
% Example~1, |\bibpunct{[}{]}{,}{a}{}{;}| changes the output of
% \begin{quote}
%   |\citep{jon90,jon91,jam92}| 
% \end{quote}
% into [Jones et al. 1990; 1991, James et al. 1992].
%
% Example~2, |\bibpunct[;]{(}{)}{,}{a}{}{;}| changes the output of
% \begin{quote}
%   |\citep[and references therein]{jon90}| 
% \end{quote}
% into (Jones et al. 1990; and references therein).
% 
% \DescribeMacro{\bibstyle@xxx}
% Usually the citation style is determined by the journal
% for which one is writing, and is as much a part of the bibliography style
% as everything else. The \thestyle{} package allows
% punctuation definitions to be directly coupled to the
% |\bibliographystyle| command that must always be present when \btx{} is
% used. It is this command that selects the \texttt{.bst} file; by adding such
% a coupling to \thestyle{} for every \texttt{.bst} file that one might
% want to use, it is not necessary to add |\bibpunct| explicitly in the
% document itself, unless of course one wishes to override the preset
% values.
% 
% Such a coupling is achieved by defining a command |\bibstyle@|{\em bst},
% where {\em bst\/} stands for the name of the \texttt{.bst} file. For example,
% the American Geophysical Union (AGU) demands in its publications that
% citations be made with square brackets and separated by semi-colons. I
% have an \texttt{agu.bst} file to accomplish most of the formatting, but such
% punctuations are not included in it. Instead, \thestyle{} has the
% definition
% \begin{quote}
% |\newcommand{\bibstyle@agu}{\bibpunct{[}{]}{;}{a}{,}{,~}}|
% \end{quote}
% 
% These style defining commands may contain more than just |\bibpunct|.
% Some numerical citation scheme require even more changes. For example,
% the journal \textsl{Nature} not only uses superscripted numbers for 
% citations, it also prints the numbers in the list of references without
% the normal square brackets. To
% accommodate this, \thestyle{} contains the style definition
% \begin{quote}\begin{verbatim}
% \newcommand{\bibstyle@nature}{\bibpunct{}{}{,}{s}{}{}%
%      \gdef\NAT@biblabelnum##1{##1.}}
% \end{verbatim}
% \end{quote}
% The redefined |\NAT@biblabelnum| command specifies how the reference numbers
% are to be formatted in the list of references itself.
% The redefinition must be made with |\gdef|, not |\def| or
% |\renewcommand|.
% 
% The selected punctuation style and other redefinitions will not be in
% effect on the first \LaTeX{} run, for they are stored to the auxiliary
% file for the subsequent run.
% 
% The user may add more such definitions of his own, to accommodate those
% journals and \texttt{.bst} files that he has. He may either add them to
% his local copy of \thestyle\texttt{.sty}, or better put them into a file
% named \thestyle\texttt{.cfg}. This file will be read in if it exists,
% adding any local configurations. Thus such configurations can survive
% future updates of the package. (This is for \LaTeXe{} only.)
% 
% \medskip
% \noindent\textbf{Note:} 
% with version~6.0, any explicit call to |\bibpunct| has priority over the 
% predefined citation styles.
% 
% \DescribeMacro{\citestyle}
% A preprogrammed citation style is normally invoked by the command
% |\bibliographystyle|, as described above. However, it may be that
% one wants to apply a certain citation style to another bibliography
% style. This may be done with |\citestyle|, given \emph{before}
% |\begin{document}|. For example, to use the
% \texttt{plain} bibliography style (for the list of references) with the
% \textsl{Nature} style of citations (superscripts),
% \begin{quote}
% |\documentclass{article}|\\
% |\usepackage{|\thestyle|}|\\
% |. . . . .|\\
% |\citestyle{nature}|\\
% |\begin{document}|\\
% |\bibliographystyle{plain}|\\
% |. . . . .|
% \end{quote}
% 
% \noindent
% \textbf{Note:} for versions 5.5 and later, all changes to the citation
% style, including punctuation, must be made before |\begin{document}|,
% which freezes the citation style.
%
% \subsection{Priority of Style Commands}\label{sec:priority}
% The citation style (punctuation and mode) can be selected by means of the
% |\bibpunct|, |\citestyle|, and predefined |\bibstyle@|\textit{bst}
% commands. They can also be selected by \LaTeXe{} options
% (Section~\ref{sec:opts}). What happens if there are several conflicting
% selections?
% 
% The lowest priority is assigned to the predefined |\bibstyle@|\textit{bst}
% commands, since they are implicit and not obvious to the user. The
% \LaTeXe{} options have the next priority. Finally, any selection by
% |\bibpunct| and/or |\citestyle| overrides those of the other methods.
% 
% \subsection{Other Formatting Options}
% 
% \DescribeMacro{\bibsection}
% The list of references normally appears as a |\section*| or |\chapter*|,
% depending on the main class. If one wants to redesign one's own heading,
% say as a numbered section with |\section|, then |\bibsection| may be
% redefined by the user accordingly.
% 
% \DescribeMacro{\bibfont}
% The list of references is normally printed in the same font size and
% style as the main body. However, it is possible to define |\bibfont|
% to be font commands that are in effect within the \texttt{thebibliography}
% environment.
%
% \DescribeMacro{\bibhang}
% The list of references for author--year styles uses a hanging indentation
% format: the first line of each reference is flush left, the following lines
% are set with an indentation from the left margin. This indentation is 1~em
% by default but may be changed by redefining (with |\setlength|) the 
% length parameter |\bibhang|.
%
% \DescribeMacro{\bibsep}
% The vertical spacing between references in the list, whether author--year
% or numerical, is controlled by the length |\bibsep|. If this is set to
% 0~pt, there is no extra line spacing between references. The default
% spacing depends on the font size selected in |\documentclass|, and is
% almost a full blank line. Change this by redefining |\bibsep| with
% |\setlength| command.
% 
% \subsection{Automatic Indexing of Citations}
% 
% (New to version 6.0; applies to \LaTeXe\ only)\\[1ex]
% \DescribeMacro{\citeindextrue}
% \DescribeMacro{\citeindexfalse}
% If one wishes to have the citations entered in the \texttt{.idx} indexing
% file, it is only necessary to issue |\citeindextrue| at any point in the
% document. All following |\cite| commands, of all variations, then insert
% the corresponding entry to that file. With |\citeindexfalse|, these
% entries will no longer be made.
% 
% The |\bibitem| commands in the \texttt{thebibliography} environment will
% also make index entries. If this is not desired, then issue
% |\citeindexfalse| before |\bibliography| or |\begin{thebibliography}|.
% 
% Of course, |\makeindex| must also be issued in the preamble to activate
% indexing, as usual. Otherwise, no indexing is done at all.
% 
% Make sure that the document has been processed at least twice after the
% last \btx\ run before running the \texttt{makeindex} program.
% 
% \DescribeMacro{\NAT@idxtxt}
% The form of the index entries is set by the internal |\NAT@idxtxt|, which
% can be redefined by hackers if wanted (in the \thestyle\texttt{.cfg} file
% please). By default, it prints the short author list plus date in the
% current parenthesis style.
% 
% The \thestyle\ package can also be used with the \texttt{index} package of
% David~M. Jones. The order in which the packages are loaded is not
% important. 
% 
% \DescribeMacro{\citeindextype}
% In that package, multiple index lists may be made by means of a |\newindex|
% command. For example, it may be desirable to put all the citation indexing
% into a separate list. First that list must be initiated with, e.g.,
% \begin{quote}
%    |\newindex{cite}{ctx}{cnd}{List of Citations}|
% \end{quote}
% and then the automatic citation indexing associated with this list with the
% \thestyle\ command
% \begin{quote}
%    |\renewcommand{\citeindextype}{cite}|
% \end{quote}
% See the documentation for \texttt{index.sty} for details.
% 
% \subsection{Hyper\TeX\ Compatibility}
% As of version~6.4, \thestyle\ is compatible with the \texttt{hyperref} 
% package of Sebastian Rahtz and Yannis Haralambous, for use with
% Hyper\TeX. The compatibility is of a mutual nature: both packages contain
% coding that interact with that of the other. Thus the version of
% \texttt{hyperref} must be more recent than 1996 Oct~8.
% ^^A Update the date when it is better known, after checking with Rahtz.
%
% \subsection{Multiple Bibliographies in One Document}\label{sec:chapbib}
% 
% As of version 6.4, \thestyle\ is compatible with the \texttt{chapterbib}
% package of Donald Arseneau and Niel Kempson,\relax
% \footnote{I have used version 1.5 from 1995/10/09; cannot guarantee 
% earlier versions.}
% which makes it possible to have several
% bibliographies in one document. The usual application is to have
% bibliographies in each chapter of a book, especially if they have been
% written by different authors.
% 
% The \texttt{chapterbib} package works in a very natural way for the
% author; only the editor who puts all the chapters together into one book
% has to do some extra work. 
% 
% The package makes use of the |\include| command, and it is in fact every
% |\include|d file that has its own bibliography. For large books, it makes
% very good sense to take advantage of this feature in any case.
% 
% To review the use of |\include|, recall that the main file
% \begin{verbatim}
%   \documentclass{...}
%   \includeonly{ch2}
%   \begin{document}
%     \include{ch1}
%     \include{ch2}
%     \include{ch3}
%   \end{document}
% \end{verbatim}
% will process only the file \texttt{ch2.tex} as though the files
% \texttt{ch1.tex} and \texttt{ch3.tex} were also present. That is, all
% counters, especially the page and section numbers, as well as
% cross-referencing definitions, will function as if the whole document
% were processed. The trick is that each |\include|d file has it own
% \texttt{.aux} file containing these definitions, and they are all read
% in every time, even if the corresponding \texttt{.tex} file is not. The
% \texttt{.aux} files also contain the citation information for \btx,
% something that the \texttt{chapterbib} package exploits.
% 
% If |\usepackage{chapterbib}| has been given, the keys in each |\cite|
% and |\bibitem| command are associated with the current |\include|d file
% and are distinguished from the identical key in a different file. Each of
% these files must contain its own |\bibliography| and |\bibliographystyle|
% commands. One processes \btx\ on each file separately before processing
% it under \LaTeX\ (at least twice).
% 
% \subsubsection{Special Considerations for \thestyle\ and
%           \texttt{chapterbib}}
% 
% The order in which the \texttt{chapterbib} and \thestyle\ packages are loaded
% is unimportant.
% 
% The \texttt{chapterbib} package provides an option \texttt{sectionbib}
% that puts the bibliography in a |\section*| instead of |\chapter*|,
% something that makes sense if there is a bibliography in each chapter.
% This option will not work when \thestyle\ is also loaded; instead, add
% the option to \thestyle. (The \texttt{sectionbib} option can always be
% given, but it only has meaning for the \texttt{book} and \texttt{report}
% classes, or for classes derived from them.)
% 
% Every |\include|d file (with citations) must contain its own
% |\bibliography| command where the bibliography is to appear. The database
% files listed as arguments to this command can be different in each file,
% of course. However, what is not so obvious, is that each file must also
% contain a |\bibliographystyle| command, \emph{preferably with the same
% style  argument}. If different bibliography styles are specified for
% different files, then the preprogrammed citation style (punctuation and
% citation mode) will be that of the first bibliography style given. The
% preprogrammed citation styles can only be changed in the preamble (see
% Section~\ref{sec:priority}), something that guarantees a uniform style
% for the entire document.\relax
%   \footnote{It would be relatively easy to allow changes in style anywhere
%   in the document, but this strikes me as bad policy. However, it is
%   provided for with the \texttt{docstrip} option \texttt{nopreonly}.}
% 
% \subsection{Sorting and Compressing Numerical Citations}
% 
% Another package by Donald Arseneau, \texttt{cite.sty}, reimplements the
% entire (numerical) citation system such that one can control the
% punctuation and citation format, all of which is done by \thestyle\ as
% well. However, it also can sort and compress numerical citations,
% something that is required by some journals. 
% 
% What this means is that when multiple citations are given with a single
% |\cite| command, the normal order of the numbers is in the sequence
% given. This is usually a wild list of numbers, such as [4,2,8,3]. With
% the \texttt{cite} package, this list becomes [2--4,8].
% 
% It is impossible to make the \texttt{cite} and \thestyle\ packages
% compatible, since both reimplement |\cite| from scratch. Instead, I have
% taken the necessary coding from \texttt{cite.sty} and modified it for
% \thestyle{}. This coding becomes activated 
% by including the option \texttt{sort} in the |\usepackage| command.
% It only affects numerical
% citations, and has no influence on the author--year ones.
% 
% \section{Numerical Citations with Author--Year Styles}\label{sec:6.0}
% 
% (New to version 6.0)\\[1ex]
% In earlier versions, each \texttt{.bst} file could be used either for
% numerical or author--year citation mode, without any possibility of
% switching. Obviously numerical \texttt{.bst} files can never be used for
% author--year citations (the information is missing in the auxiliary file)
% but there is no reason why the converse should not work. 
% 
% I have been frequently asked if I could implement this feature, and I at
% first replied that it would be quite easy. That was a mistake. The
% numerical information was missing in the auxiliary files, so the
% |\cite| commands could never access it.
% 
% \subsection{New Coding}
% I have now recoded \thestyle{} so that the separation of author and year
% information comes before the information is written to the auxiliary
% file, and furthermore, a sequential number is also written. This permits
% numerical citations with \emph{any} of the author--year bibliographic style
% files.
% 
% The recoding has made fewer internal changes than I expected. Any users
% who have local redefinitions of my internals should be aware that of the
% citation commands, only |\NAT@citexnum| has been altered. Otherwise the
% changes are in the parsing commands |\NAT@parse| and |\NAT@parse@date|.
% Of the external commands, |\bibitem| is completely redefined, and
% |\citeauthor|, |\citeyear|, and |\citefullauthor| are different.
% 
% Version~5.5 actually made far more internal changes: all internal
% commands were renamed to conform to recommended \LaTeX{} coding
% practices.
% 
% \subsection{Selecting Numerical Mode}
% By default, \thestyle{} is in author--year mode. This can be changed by
% \begin{enumerate}
%   \item selecting a numerical bibliography style with predefined
%         citation style, defined either in the package or in the local
%         configuration file;
%         
%   \item giving options \texttt{numbers} or \texttt{super} to the
%         |\usepackage| command;
% 
%   \item issuing |\bibpunct| with the 4th mandatory argument set to \texttt{n}
%         or \texttt{s};
% 
%   \item issuing |\citestyle| with the name of a predefined numerical
%         bibliography style.
% \end{enumerate}
% The methods are listed in order of increasing priority. 
% 
% The \thestyle{} package will automatically switch to numerical mode if
% any one of the |\bibitem| entries fails to conform to the possible
% author--year formats. There is no way to override this, since such an
% entry would cause trouble in the author--year mode.
% 
% There are certain special `numerical' styles, like that of the standard
% \texttt{alpha.bst}, which include a non-numerical label in place of the
% number, in the form
% \begin{quote} |\bibitem[ABC95]{able95}| \end{quote}
% As far as \thestyle\ is concerned, this label does not conform to the
% author--year possibilities and is therefore considered to be numerical.
% The citation mode switches to numerical, and |\cite{able95}| prints
% [ABC95].
% 
% \subsection{New Bibliography Styles}
% I provided three new \texttt{.bst} files to replace the standard \LaTeX\
% numerical ones:
% \begin{quote}\ttfamily
%  plainnat.bst \qquad abbrvnat.bst \qquad unsrtnat.bst
% \end{quote}
% These produce reference lists in the same style as the corresponding
% standard \texttt{.bst} file, but can only be used with \thestyle{}. The
% advantage is that they can be used in both numerical and author--year
% mode.
% 
% In fact, \texttt{unsrtnat.bst} is rather silly for author--year
% citations: the references are listed in the order they were cited, which
% makes it difficult to find them. It is only included in case an author
% wishes to switch from \texttt{plainnat} (author--year) to the unsorted
% numerical mode, retaining the |\citeauthor| and |\citeyear| commands in
% his paper.
% 
% It is not possible to produce a modified version of \texttt{alpha.bst}
% since it would conflict with the |\bibitem| format of \thestyle.
% 
% The sample bibliography style file \texttt{natbib.bst} will no longer be
% distributed. It merely represented a demonstration of the |\bibitem|
% syntax for \thestyle, a function that is now taken over by
% \texttt{plainnat.bst} and friends.
% 
% \section{Local Configuration}
% For \LaTeXe, it is possible to add a local configuration file
% \thestyle\texttt{.cfg}, which is read in, if it exists, at
% the end of the package. It may thus contain coding to supecede that in
% the package, although its main purpose is to allow the user to add his
% own |\bibstyle@|\textit{bst} definitions to couple citation punctuation
% with local bibliography styles.
% 
% \section{Options with \LaTeXe}\label{sec:opts}
% One of the new features of \LaTeXe{} is \emph{options} for the packages,
% in the same way as main styles (now called \emph{classes}) can take
% options. This package is now installed with
% \begin{quote}
% |\documentclass[..]{...}|\\
% |\usepackage[|\emph{options}|]{|\thestyle|}|
% \end{quote}
% The options available provide another means of specifying the
% punctuation for citations:
% \begin{description}
% \item[\ttfamily round] (default) for round parentheses;
% \item[\ttfamily square] for square brackets;
% \item[\ttfamily curly] for curly braces;
% \item[\ttfamily angle] for angle brackets;
% \item[\ttfamily colon] (default) to separate multiple citations with
%      colons;
% \item[\ttfamily comma] to use commas as separaters;
% \item[\ttfamily authoryear] (default) for author--year citations;
% \item[\ttfamily numbers] for numerical citations;
% \item[\ttfamily super] for superscripted numerical citations, as in
%      \textsl{Nature};
% \item[\ttfamily sort] puts multiple numerical citations in order and
%      compresses them if possible (as 3--6, 15);
% \item[\ttfamily sectionbib] redefines |\thebibliography| to issue
%      |\section*| instead of |\chapter*|; valid only for classes with a
%      |\chapter| command; to be used with the \texttt{chapterbib} package.
% \end{description}
% 
% If any of these options are selected, the predefined citation styles in
% the commands |\bibstyle@|\textit{bst} will be no longer be effective.
% If either |\bibpunct| or |\citestyle| is given in the preamble, the above
% punctuation options will no longer hold.
% 
% \section{As Module to Journal-Specific Styles}
% Although \thestyle{} is meant to be an all-purpose bibliographic style
% \emph{package}, it may also be incorporated as a module to other 
% packages for specific journals. In this case, many of the general features may
% be left off. This is allowed for with \texttt{docstrip} options that not
% only leave off certain codelines, but also include extra ones. So far,
% options exist for 
% \begin{description}
% \item[\ttfamily subpack] produces a basic version with author--year only,
% fixed citation punctuation, no |\bibpunct| nor |\citestyle| nor 
% predefined styles;
% \item[\ttfamily subpack,egs] for journals of the \textsl{European Geophysical
% Society}, in particular \textsl{Nonlinear Processes in Geophysics};
% \item[\ttfamily subpack,agu] for \textsl{American Geophysical Union} journals.
% \end{description}
% The \texttt{subpack} option must always be used with \texttt{package}.
% 
% Previous options \texttt{jgr} and \texttt{grl} have become obsolete due
% to revisions in these journals; they have been replaced by the more
% general \texttt{agu} option.
% 
% \section{Summary}
% 
% The following summary is included as comments at the beginning of the
% \thestyle\texttt{.sty} file, for quick reference. Certain lines are
% dependent on the \texttt{docstrip} options.
% 
% \iffalse
%<*package&!subpack>
% \fi
% \begin{verbatim}
 % This package reimplements the LaTeX \cite command to be used for various
 % citation styles, both author-year and numerical. It accepts BibTeX
 % output intended for many other packages, and therefore acts as a 
 % general, all-purpose citation-style interface.
 %
 % With standard numerical .bst files, only numerical citations are
 % possible. With an author-year .bst file, both numerical and
 % author-year citations are possible.
 %
 % If author-year citations are selected, \bibitem must have one of the
 %   following forms:
 %   \bibitem[Jones et al.(1990)]{key}...
 %   \bibitem[Jones et al.(1990)Jones, Baker, and Williams]{key}...
%<*apalike|all>
 %   \bibitem[Jones et al., 1990]{key}...
%</apalike|all>
%<*newapa|chicago|all>
 %   \bibitem[\protect\citeauthoryear{Jones, Baker, and Williams}{Jones
 %       et al.}{1990}]{key}...
 %   \bibitem[\protect\citeauthoryear{Jones et al.}{1990}]{key}...
%</newapa|chicago|all>
%<*astron|all>
 %   \bibitem[\protect\astroncite{Jones et al.}{1990}]{key}...
%</astron|all>
%<*authordate|all>
 %   \bibitem[\protect\citename{Jones et al., }1990]{key}...
%</authordate|all>
%<*harvard|all>
 %   \harvarditem[Jones et al.]{Jones, Baker, and Williams}{1990}{key}...
%</harvard|all>
 %   
 % This is either to be made up manually, or to be generated by an 
 % appropriate .bst file with BibTeX.
 %                            Author-year mode     ||   Numerical mode
 % Then, \citet{key}  ==>>  Jones et al. (1990)    ||   Jones et al. [21]
 %       \citep{key}  ==>> (Jones et al., 1990)    ||   [21]
 % Multiple citations as normal:
 % \citep{key1,key2}  ==>> (Jones et al., 1990; Smith, 1989) || [21,24]
 %                           or  (Jones et al., 1990, 1991)  || [21,24]
 %                           or  (Jones et al., 1990a,b)     || [21,24]
 % \cite{key} is the equivalent of \citet{key} in author-year mode
 %                         and  of \citep{key} in numerical mode
 % Full author lists may be forced with \citet* or \citep*, e.g.
 %       \citep*{key}      ==>> (Jones, Baker, and Williams, 1990)
 % Optional notes as:
 %   \citep[chap. 2]{key}    ==>> (Jones et al., 1990, chap. 2)
 %   \citep[e.g.,][]{key}    ==>> (e.g., Jones et al., 1990)
 %   \citep[see][pg. 34]{key}==>> (see Jones et al., 1990, pg. 34)
 %  (Note: in standard LaTeX, only one note is allowed, after the ref.
 %   Here, one note is like the standard, two make pre- and post-notes.)
 %   \citealt{key}          ==>> Jones et al. 1990
 %   \citealt*{key}         ==>> Jones, Baker, and Williams 1990
 %   \citealt{key1,key2}    ==>> Jones et al. 1990; Smith 1989
 % Additional citation possibilities (both author-year and numerical modes)
 %   \citeauthor{key}     ==>> Jones et al.
 %   \citeyear{key}       ==>> 1990
 %   \citefullauthor{key} ==>> Jones, Baker, and Williams
 % (Multiple keys NOT allowed!)
 % Note: full author lists depends on whether the bib style supports them;
 %       if not, the abbreviated list is printed even when full requested.
 %
 % Defining the citation style of a given bib style:
%<!nopreonly> % Use \bibpunct (in the preamble only) with 6 mandatory arguments:
%<nopreonly> % Use \bibpunct (anywhere in the text) with 6 mandatory arguments:
 %    1. opening bracket for citation
 %    2. closing bracket
 %    3. citation separator (for multiple citations in one \cite)
 %    4. the letter n for numerical styles, s for superscripts
 %        else anything for author-year
 %    5. punctuation between authors and date
 %    6. punctuation between years when common authors missing
 % One optional argument is the character coming before post-notes. It
 %   appears in square braces before all other arguments. May be left off.
 % Example (and default) \bibpunct[,]{(}{)}{;}{a}{,}{,}
 % 
 % To make this automatic for a given bib style, named newbib, say, make
 % a local configuration file, natbib.cfg, with the definition
 %   \newcommand{\bibstyle@newbib}{\bibpunct...}
 % Then the \bibliographystyle{newbib} will cause \bibstyle@newbib to 
 % be called on THE NEXT LATEX RUN (via the aux file).
 %
%<!nopreonly> % Such preprogrammed definitions may be invoked in the text (preamble only)
%<nopreonly> % Such preprogrammed definitions may be invoked anywhere in the text
 %  by calling \citestyle{newbib}. This is only useful if the style specified
 %  differs from that in \bibliographystyle.
 %
%<*!209>
 % With \citeindextrue and \citeindexfalse, one can control whether the
 % \cite commands make an automatic entry of the citation in the .idx
 % indexing file. For this, \makeindex must also be given in the preamble.
 %
 % LaTeX2e Options: (for selecting punctuation)
 %   round  -  round parentheses are used (default)
 %   square -  square brackets are used   [option]
 %   curly  -  curly braces are used      {option}
 %   angle  -  angle brackets are used    <option>
 %   colon  -  multiple citations separated by colon (default)
 %   comma  -  separated by comma
 %   authoryear - selects author-year citations (default)
 %   numbers-  selects numerical citations
 %   super  -  numerical citations as superscripts
 %   sort   -  sorts and compresses numerical citations 
 %   sectionbib - puts bibliography in a \section* instead of \chapter*
 % Punctuation so selected dominates over any predefined ones.
 % LaTeX2e options are called as, e.g.
 %        \usepackage[square,comma]{natbib}
%</!209>
% \end{verbatim}
% \iffalse
 % LaTeX the source file natbib.dtx to obtain more details.
 %-----------------------------------------------------------
%</package&!subpack>
% \fi
% 
% \section{Options with \texttt{docstrip}}
% The source \texttt{.dtx} file is meant to be processed with
% \texttt{docstrip}, for which a number of options are available:
% \begin{description}
% \item[\ttfamily all] includes all of the other interfaces;
%
% \item[\ttfamily apalike] allows interpretation of minimal \texttt{apalike} 
%    form of |\bibitem|;
%
% \item[\ttfamily newapa] allows |\citeauthoryear| to be in the optional 
%    argument of |\bibitem| along with the punctuation commands of 
%    \texttt{newapa.sty};
%
% \item[\ttfamily chicago] is the same as \texttt{newapa};
%
% \item[\ttfamily harvard] includes interpretation of |\harvarditem|;
%
% \item[\ttfamily astron] allows |\astroncite| to appear in the optional 
%    argument of |\bibitem|;
%
% \item[\ttfamily authordate] adds the syntax of the |\citename| command.
%
% \end{description}
%
% This package file is intended to act as a module for other class files
% written for specific journals, in which case the flexible
% |\bibstyle@|{\em bst\/} commands are not wanted. Punctuation and
% other style features are to be rigidly fixed. These journal options are
% \begin{description}
% \item[\ttfamily agu] for journals of the \textsl{American Geophysical
%   Union};
%
% \item[\ttfamily egs] for journals of the \textsl{European Geophysical 
%   Society}, in particular \textsl{Nonlinear Processes in Geophysics}.
%
% \end{description}
%
% The remaining options are:
% \begin{description}
% \item[\ttfamily package] to produce a \texttt{.sty} package file with most
%     comments removed;
% 
% \item[\ttfamily 209] (together with \texttt{package}) for a style option 
%     file that will run under the older \LaTeX~2.09;
% 
% \item[\ttfamily subpack] (together with \texttt{package}) for coding that 
%     is to be included inside a larger package; even more comments are 
%     removed, as well as \LaTeXe{} option handling and identification;
%     produces a basic \thestyle\ package for author--year only, fixed
%     citation style (punctuation);
%
% \item[\ttfamily nopreonly] allows |\citestyle| and |\bibpunct| to be
%     called anywhere in the text; this is considered possibly useful
%     with the \texttt{chapterbib} package where different chapters might 
%     have different bibliography and citation styles; is only provided 
%     in case I change my mind about this feature, but for now I 
%     refuse to implement it;
%
% \item[\ttfamily driver] to produce a driver \texttt{.drv} file that will
%     print out the documentation under \LaTeXe. The documentation cannot
%     be printed under \LaTeX~2.09.
% 
% \end{description}
% The source file \texttt{\filename.dtx} is itself a driver file and can
% be processed directly by \LaTeXe.
%
% \StopEventually{\PrintIndex\PrintChanges}
% 
% \section{The Coding}
% This section presents and explains the actual coding of the macros.
% It is nested between |%<*package>| and |%</package>|, which
% are indicators to \texttt{docstrip} that this coding belongs to the package 
% file.
%
% The \texttt{docstrip} option |<subpack>| should only be called if the
% coding is to be included as part of another package, in which case the 
% announcement text and \LaTeXe{} options are suppressed.
%
% An alternative version of this coding is provided for running as a 
% style file under \LaTeX~2.09. Code lines belonging to this are
% indicated with guard |<209>|; those for LaTeXe{} only with |<!209>|.
%
% With \LaTeX\ release from \texttt{1995/12/01}, |\cite| is made robust, 
% something that I have adopted as well.
% 
% Another change in the above release is that 
% |\G@refundefinedtrue|\footnote{In fact, it was only proposed to be 
%    removed, but because several packages, including this one, were 
%    adversely affected, the name was retained for the new coding.}
% and the |\if@openbib| flag have been removed.
% New, more effective, and compact, coding has replaced them. Rather than
% relying on the kernel coding, I make \thestyle\ more
% self-reliant. 
% 
% \subsection{Test for Specific Journals}
% This package should not be loaded with the specific journal packages that
% already include it; it is superfluous and can lead to contradictions.
%    \begin{macrocode}
%<*package>
%<*!209&!subpack>
\@ifclassloaded{aguplus}{\PackageError{natbib}
  {The aguplus class already includes natbib coding,\MessageBreak
   so you should not add it explicitly}
  {Type <Return> for now, but then later remove\MessageBreak
   the command \protect\usepackage{natbib} from the document}
  \endinput}{}
\@ifclassloaded{nlinproc}{\PackageError{natbib}
  {The nlinproc class already includes natbib coding,\MessageBreak
   so you should not add it explicitly}
  {Type <Return> for now, but then later remove\MessageBreak
   the command \protect\usepackage{natbib} from the document}
  \endinput}{}
\@ifclassloaded{egs}{\PackageError{natbib}
  {The egs class already includes natbib coding,\MessageBreak
   so you should not add it explicitly}
  {Type <Return> for now, but then later remove\MessageBreak
   the command \protect\usepackage{natbib} from the document}
  \endinput}{}
%</!209&!subpack>
%    \end{macrocode}
%
% \subsection{Selecting Citation Punctuation and Other Modifications}
% \begin{macro}{\bibstyle@xxx}
% \changes{5.3}{1994 Sep 13}{Add \texttt{agsm} and \texttt{dcu} punctuation 
%    styles from \texttt{harvard} series}
% We begin by defining a number of punctuation styles for specific
% author--year \texttt{.bst} files that I use. 
% They are placed here, near the beginning, so that another
% user can easily find them to add his own. Some comments remain in the
% stripped version too.
%    \begin{macrocode}
%<*!subpack>
 % Define citation punctuation for some author-year styles
 % One may add and delete at this point
%<!209> % Or put additions into local configuration file natbib.cfg
\newcommand\bibstyle@chicago{\bibpunct{(}{)}{;}{a}{,}{,}}
\newcommand\bibstyle@named{\bibpunct{[}{]}{;}{a}{,}{,}}
\newcommand\bibstyle@agu{\bibpunct{[}{]}{;}{a}{,}{,~}}%Amer. Geophys. Union
\newcommand\bibstyle@egs{\bibpunct{(}{)}{;}{a}{,}{,}}%Eur. Geophys. Soc.
%<*all|harvard>
\newcommand\bibstyle@agsm{\bibpunct{(}{)}{,}{a}{}{,}\gdef\harvardand{\&}}
\newcommand\bibstyle@kluwer{\bibpunct{(}{)}{,}{a}{}{,}\gdef\harvardand{\&}}
\newcommand\bibstyle@dcu{\bibpunct{(}{)}{;}{a}{;}{,}\gdef\harvardand{and}}
%</all|harvard>
%    \end{macrocode}
% 
% Here give the explicit punctuation for the specific journals; their
% coding need not contain the punctuation macros |\bibpunct| and
% |\bibstyle@xxx| since the punctuation is fixed for them.
%    \begin{macrocode}
\newcommand\bibstyle@aa{\bibpunct{(}{)}{;}{a}{}{,}} %Astronomy & Astrophysics
\newcommand\bibstyle@pass{\bibpunct{(}{)}{;}{a}{,}{,}}%Planet. & Space Sci
\newcommand\bibstyle@anngeo{\bibpunct{(}{)}{;}{a}{,}{,}}%Annales Geophysicae
\newcommand\bibstyle@nlinproc{\bibpunct{(}{)}{;}{a}{,}{,}}%Nonlin.Proc.Geophys.
%</!subpack>
%<agu>\newcommand\NAT@open{[} \newcommand\NAT@close{]} 
%<agu>\newcommand\NAT@sep{;} \newcommand\NAT@cmt{,}
%<agu>\newcommand\NAT@aysep{,} \newcommand\NAT@yrsep{,~}
%<egs>\newcommand\NAT@open{(} \newcommand\NAT@close{)} 
%<egs>\newcommand\NAT@sep{;} \newcommand\NAT@cmt{,}
%<egs>\newcommand\NAT@aysep{,} \newcommand\NAT@yrsep{,}
%    \end{macrocode}
% 
% \changes{6.5}{1997 Jan 30}{Fix \texttt{esa} style for new note system}
% Next, the same thing is done for some numerical styles. A major
% difference here is that the |\@biblabel| and |\@cite| commands must also
% be redefined in many cases. These redefinitions must be made with the
% |\gdef| (global definition) command.
%    \begin{macrocode}
%<*!subpack>
 % Define citation punctuation for some numerical styles
\newcommand\bibstyle@cospar{\bibpunct{/}{/}{,}{n}{}{}%
     \gdef\NAT@biblabelnum##1{##1.}}
\newcommand\bibstyle@esa{\bibpunct{(}{)}{,}{n}{}{}%
     \gdef\NAT@biblabelnum##1{##1.\hspace{1em}}%
     \gdef\NAT@citenum##1##2##3{\NAT@open \if\relax##2\relax\else
       ##2\ \fi Ref.~##1\if\relax##3\relax\else\NAT@cmt\ ##3\fi\NAT@close}}
\newcommand\bibstyle@nature{\bibpunct{}{}{,}{s}{}{}%
     \gdef\NAT@biblabelnum##1{##1.}}
%    \end{macrocode}
% 
% Finally, the standard \LaTeX{} (numerical) citation styles are included
%  along with their \thestyle\ modifications (author--year).
%    \begin{macrocode}
 % The standard LaTeX styles
\newcommand\bibstyle@plain{\bibpunct{[}{]}{,}{n}{}{}}
\let\bibstyle@alpha=\bibstyle@plain
\let\bibstyle@abbrv=\bibstyle@plain
\let\bibstyle@unsrt=\bibstyle@plain
 % The author-year modifications of the standard styles
\newcommand\bibstyle@plainnat{\bibpunct{[}{]}{,}{a}{,}{,}}
\let\bibstyle@abbrvnat=\bibstyle@plainnat
\let\bibstyle@unsrtnat=\bibstyle@plainnat
%    \end{macrocode}
% \end{macro}
% 
% \subsection{\LaTeXe{} Options}
% \begin{macro}{\ifNAT@numbers}
% \changes{5.5}{1995 Mar 21}{Add flag for switching between cite styles}
% \begin{macro}{\ifNAT@super}
% \changes{5.5}{1995 Mar 21}{Add flag for switching between cite styles}
% Two flags are used to keep track of the citation style. The flag
% |\ifNAT@numbers| is \meta{false} by default (author--year) and switches
% to \meta{true} for numerical citations. The flag |\ifNAT@super| indicates
% if the numerical citation is to be done with superscripts.
%    \begin{macrocode}
\newif\ifNAT@numbers \NAT@numbersfalse
\newif\ifNAT@super \NAT@superfalse
%</!subpack>
%    \end{macrocode}
% \end{macro}\end{macro}
%
% \begin{macro}{\DeclareOption}
% \changes{5.0}{1994 May 18}{Add \LaTeXe{} options.}
% \changes{5.3}{1994 Sep 13}{Add options \texttt{angle} and \texttt{curly}}
% \changes{6.0}{1995 Sep 6}{Add option \texttt{bibstyle}}
% \changes{6.0}{1995 Sep 6}{Execute \texttt{nobibstyle} with all options}
% \changes{6.1}{1995 Nov 27}{Add option \texttt{openbib}}
% \changes{6.4}{1996 Sep 1}{Add option \texttt{sectionbib}}
% \changes{6.4}{1996 Sep 13}{Add option \texttt{sort}}
% For \LaTeXe, we can define some options to be used with the |\usepackage|
% command that loads the \thestyle{} package. These are an additional means
% of specifying the citation punctuation and scheme.
% 
% The standard option \texttt{openbib} is also defined here so that its
% behaviour is entirely under my control. Relying on the standard classes
% can lead to trouble.
%    \begin{macrocode}
%<*!subpack&!209>
\DeclareOption{numbers}{\NAT@numberstrue
   \ExecuteOptions{square,comma,nobibstyle}}
\DeclareOption{super}{\NAT@supertrue\NAT@numberstrue
   \ExecuteOptions{nobibstyle}}
\DeclareOption{authoryear}{\NAT@numbersfalse
   \ExecuteOptions{round,colon,bibstyle}}
\DeclareOption{round}{%
      \renewcommand\NAT@open{(} \renewcommand\NAT@close{)}
   \ExecuteOptions{nobibstyle}}
\DeclareOption{square}{%
      \renewcommand\NAT@open{[} \renewcommand\NAT@close{]}
   \ExecuteOptions{nobibstyle}}
\DeclareOption{angle}{%
      \renewcommand\NAT@open{$<$} \renewcommand\NAT@close{$>$}
   \ExecuteOptions{nobibstyle}}
\DeclareOption{curly}{%
      \renewcommand\NAT@open{\{} \renewcommand\NAT@close{\}}
   \ExecuteOptions{nobibstyle}}
\DeclareOption{comma}{\renewcommand\NAT@sep{,}
   \ExecuteOptions{nobibstyle}}
\DeclareOption{colon}{\renewcommand\NAT@sep{;}
   \ExecuteOptions{nobibstyle}}
\DeclareOption{nobibstyle}{\let\bibstyle=\@gobble}
\DeclareOption{bibstyle}{\let\bibstyle=\@citestyle}
\newif\ifNAT@openbib \NAT@openbibfalse
\DeclareOption{openbib}{\NAT@openbibtrue}
\DeclareOption{sectionbib}{\def\NAT@sectionbib{on}}
\def\NAT@sort{0}
\DeclareOption{sort}{\def\NAT@sort{1}}
\@ifpackageloaded{cite}{\PackageWarningNoLine{natbib}
  {The `cite' package should not be used\MessageBreak
   with natbib. Use option `sort' instead}\ExecuteOptions{sort}}{}
%</!subpack&!209>
%    \end{macrocode}
% Before version~6.0, the punctuation options had the lowest priority,
% being overridden by any |\bibstyle@|\textit{bst} command. It was necessary
% to add the option \texttt{nobibstyle} to make them effective. Now this
% option is automatically executed whenever a punctuation option is chosen.
% The reason is that \texttt{nobibstyle} is either superfluous (no
% predefined citation style is present) or always necessary to make the
% other options meaningful. 
% 
% It seems to be better practice to make any explicit choices to have
% higher priority over implicit ones.
% 
% The option \texttt{bibstyle} reactivates the predefined commands, just in
% case this is needed, as it is for the option \texttt{authoryear}. This
% leaves the predefined styles functioning, since it does not really
% determine the citation style, but rather establishes a default which may
% be overwritten by a prestored one.  For this reason, this option must be
% defined very early so that any explicit punctuation options overwrite the
% defaults. (Options are executed in the order defined.)
% 
% \end{macro}
%
% \subsection{Selecting Citation Punctuation and Other Modifications}
% \begin{macro}{\bibstyle}
% \changes{5.4}{1994 Nov 24}{Move \cs{ProcessOptions} to after definition
%    of \cs{bibstyle}}
% The pre-defined punctuation styles are associated with particular
% \texttt{.bst} files. This is implemented by means of the standard
% |\bibliographystyle| command that specifies the name of the \texttt{.bst}
% file that is to be used by \btx. Some \LaTeX{} manuals erroneously state
% that this command must be issued prior to any |\cite| commands. If that
% were the case, the task of invoking the style definitions would be much
% easier, for they certainly must be issued before the first |\cite|.
% However, I have always called |\bibliographystyle| together with
% |\bibliography|, well after all |\cite| commands. 
% 
% In fact, all that |\bibliographystyle|\marg{bst} does is to write
% |\bibstyle|\marg{bst} to the auxiliary file. This file, and all its
% contents, will be read in at the beginning of the next \LaTeX{} run; to
% be precise, it is read in by |\begin{document}|. The command |\bibstyle| is
% then executed at that point on the next run. However, it is 
% defined to do nothing more than to swallow up its argument! In other
% words, the combination |\bibliographystyle| and |\bibstyle| do absolutely
% nothing to the \LaTeX{} runs. In fact, |\bibstyle| is a command for
% \btx{} alone.
% 
% I have taken advantage of this to redefine |\bibstyle| to execute the
% \texttt{bst}-specific definitions. This definition is made here before
% |\ExecuteOptions| so that options may make use of it; the option
% \texttt{nobibstyle} turns it off, for example.
%    \begin{macrocode}
%<*!subpack>
\renewcommand\bibstyle[1]{\@ifundefined{bibstyle@#1}{\relax}
     {\csname bibstyle@#1\endcsname}}
%    \end{macrocode}
% This is executed only when the auxiliary file is read in, and that is
% when |\begin{document}| is issued. Thus this is well before any |\cite|
% commands. It is for this reason that any definitions in the
% |\bibstyle@xxx| commands must be global.
% 
% A minor problem arises in that the auxiliary file is read in a second
% time, at the end of the run, to check whether any labels have changed. 
% Sometimes, re-executing the |\bibstyle@xxx| command at this point
% produces error messages. To avoid this, |\bibstyle| is reset to its
% normal, do-nothing, definition after |\begin{document}|.
% 
% \LaTeXe{} provides a better way of adding commands to |\document|,
% which is then exploited here. Paradoxically, the problem with re-reading
% the auxiliary file does not seem to arise for this version.
% 
% (|\NAT@set@cites| must also be executed at the start of the document; for
% \LaTeXe, this is added later when the macro is defined.)
%    \begin{macrocode}
%<*209>
\let\ori@document=\document
\renewcommand\document{\ori@document\global\let\bibstyle=\@gobble 
  \NAT@set@cites}
%</209>
%<!209>\AtBeginDocument{\global\let\bibstyle=\@gobble}
%    \end{macrocode}
% \end{macro}
% 
% \begin{macro}{\citestyle}
% \changes{5.4}{1995 Feb 03}{Add macro}
% \changes{5.5}{1995 Mar 24}{Make preamble only}
% To allow one to invoke preprogrammed punctuation styles that are
% different from the name of the bibliography style specified by
% |\bibliographystyle|, call |\citestyle| in the preamble. This then
% deactivates |\bibstyle| so that any preprogrammed punctuations become
% ineffective.
% 
% This, and all other punctuation (citation style) declarations, may only
% be used in the preamble, as of version 5.5. (Previously they could only
% be used after the preamble, but the preamble is more logical.)
%    \begin{macrocode}
\let\@citestyle\bibstyle
\newcommand\citestyle[1]{\@citestyle{#1}\let\bibstyle\@gobble}
%<!209&!nopreonly>\@onlypreamble{\citestyle}\@onlypreamble{\@citestyle}
%    \end{macrocode}
% \end{macro}
%
% \begin{macro}{\bibpunct}
% \changes{5.4}{1994 Nov 24}{Add superscript style of citations}
% \changes{5.5}{1995 Mar 21}{Use flags to select citation style}
% \changes{5.5}{1995 Mar 24}{Make preamble only}
% \changes{6.0}{1995 Sep 6}{Sets \cs{bibstyle} to \cs{@gobble}}
% \changes{6.0}{1995 Sep 6}{Declare \cs{NAT@numbersfalse} before testing 4th
%      argument}
% \changes{6.0}{1995 Sep 19}{Add optional argument for \cs{NAT@cmt}}
% Now |\bibpunct| is defined. It sets various punctuation commands equal to
% its arguments. It also switches to numerical style if that is specified
% by the fourth argument. This is done by setting the citation style
% flag |NAT@numbers| accordingly. The internal cite macros are later
% set to the style-dependent definitions by calling |\NAT@set@cites|.
% 
% The fourth argument can also be \texttt{s} for superscript, which
% sets both the numerical and the |\NAT@super| flags.
% 
% An optional argument has been added to determine the punctuation that
% precedes a trailing note. This makes the counting of mandatory arguments
% somewhat confused, since the fourth argument is now |#5|.
% 
%    \begin{macrocode}
%<!209>\newcommand\bibpunct[7][,]%
%<209>\newcommand\bibpunct{\@ifnextchar[{\@bibpunct}{\@bibpunct[,]}}
%<209>\def\@bibpunct[#1]#2#3#4#5#6#7%
  {\gdef\NAT@open{#2}\gdef\NAT@close{#3}\gdef
   \NAT@sep{#4}\global\NAT@numbersfalse\ifx #5n\global\NAT@numberstrue
   \else
   \ifx #5s\global\NAT@numberstrue\global\NAT@supertrue
   \fi\fi
   \gdef\NAT@aysep{#6}\gdef\NAT@yrsep{#7}%
   \gdef\NAT@cmt{#1}%
   \global\let\bibstyle\@gobble
%<nopreonly>   \NAT@set@cites
  }
%<*!nopreonly>
%<!209>\@onlypreamble{\bibpunct}
%<209>\def\do{\noexpand\do\noexpand}
%<209>\edef\@preamblecmds{\@preamblecmds \do\bibpunct \do\citestyle}
%</!nopreonly>
%</!subpack>
%    \end{macrocode}
% The command turns off |\bibstyle| so that the predefined
% |\bibstyle@|\textit{bst} commands no longer work. Thus |\bibpunct| has
% priority over the predefined styles.
% \end{macro}
% 
% \subsection{Setting the Defaults}
% For \LaTeXe, the defaults (punctuation and scheme type) are set with the
% |\ExecuteOptions| command. For \LaTeX~2.09, the default scheme is
% author--year, and punctuation can only be set with |\bibpunct|.
%
% \begin{macro}{\NAT@open}
% \changes{5.5}{1995 May 14}{Renamed from \cs{@citebegin}}
% \begin{macro}{\NAT@close}
% \changes{5.5}{1995 May 14}{Renamed from \cs{@citeend}}
% \begin{macro}{\NAT@sep}
% \changes{5.5}{1995 May 14}{Renamed from \cs{@citesep}}
% \begin{macro}{\NAT@aysep}
% \changes{5.5}{1995 May 14}{Renamed from \cs{@auyrsep}}
% \begin{macro}{\NAT@yrsep}
% \changes{5.5}{1995 May 14}{Renamed from \cs{@yrsep}}
% \begin{macro}{\NAT@cmt}
% \changes{6.0}{1995 Sep 19}{Add macro}
% The five punctuation characters for citations are the start and stop
% brackets, the character between multiple citations, the character between
% the authors and year (parenthetical only), and the character between
% adjacent years when the common author list is omitted. 
% A sixth one has been added: the character preceding post-notes.
% Define their default values. 
%    \begin{macrocode}
%<!agu&!egs>\newcommand\NAT@open{(} \newcommand\NAT@close{)} 
%<!agu&!egs>\newcommand\NAT@sep{;} 
%    \end{macrocode}
% \end{macro}\end{macro}\end{macro}\end{macro}\end{macro}\end{macro}
% \changes{6.0}{1995 Sep 19}{Remove \cs{ExecuteOptions} since default
%        style determined by definitions}
% The option \texttt{authoryear} was originally provided to serve as the
% default; hence its definition includes \texttt{bibstyle}. However, this
% is not necessary since all the appropriate parameters are set to these
% values anyway. (One could consider removing \texttt{bibstyle}.)
%    \begin{macrocode}
%<*!subpack&!209>
%\ExecuteOptions{authoryear}
\ProcessOptions
%</!subpack&!209>
%<!agu&!egs>\newcommand\NAT@aysep{,} \newcommand\NAT@yrsep{,}
%<!agu&!egs>\newcommand\NAT@cmt{,}
%    \end{macrocode}
%
% \subsection{Internal Citing Macros}
% A number of internal macros (|\@citex|, |\@cite|, |\@biblabel|, and
% |\@bibsetup|) need to have different definitions for author--year and
% numerical schemes. They are defined here for the two different styles,
% with different names, and which ones become active depend on the citation
% style flags |\ifNAT@numbers| and |\ifNAT@super|.
% 
% The coding for the journals always defines the internal macros directly
% as the author--year version, since they can never do any switching.
% 
% \begin{macro}{\@cite}
% \changes{5.5}{1995 Mar 21}{Is \cs{let} equal to the numerical or
%      author--year definition}
% \begin{macro}{\NAT@cite}
% \changes{4.2}{1993 Dec 2}{Reversed optional text no longer needs to include
%       a trailing blank}
% \changes{5.0}{1994 May 18}{Add optional notes after as well as before
%      citation.}
% \changes{5.5}{1995 Mar 13}{Add command space in place of space}
% \changes{5.5}{1995 Mar 13}{Use \cs{if}\cs{relax}\texttt{\#2}\cs{relax}
%    in place of \cs{if}\texttt{\#2}\cs{@empty}}
% \changes{5.5}{1995 Mar 21}{Renamed from \cs{@cite}}
% \begin{macro}{\NAT@citenum}
% \changes{5.0}{1994 May 18}{Add optional notes after as well as before
%      citation.}
% \changes{5.5}{1995 Mar 13}{Add command space in place of space}
% \changes{5.5}{1995 Mar 13}{Use \cs{if}\cs{relax}\texttt{\#2}\cs{relax}
%    in place of \cs{if}\texttt{\#2}\cs{@empty}}
% \changes{5.5}{1995 Mar 21}{Renamed from \cs{@citenum}}
% \begin{macro}{\NAT@citesuper}
% \changes{5.4}{1994 Nov 24}{Add macro for superscripts}
% \changes{5.5}{1995 Mar 21}{Renamed from \cs{@citesuper}}
% \changes{5.5}{1995 Mar 27}{Prints post-note only if present}
% \changes{6.2}{1995 Feb 2}{Get superscript font size right; use
%       \cs{textsuperscript}}
% \changes{6.5}{1997 Jan 30}{Numericals always print notes if present, do not
%    test for any flag first}
% Define the internal |\@cite| commands that prints the assembled string
% of citation label texts, plus possible optional notes, before and after.
% The numerical version is almost the same, except that the brackets
% |\NAT@open| and |\NAT@close| are always present and the optional notes are
% included if they are present. For author-year, the notes are included (if
% present) here only for parenthetical citations (|\ifNAT@swa| if \meta{true})
% and for textual citations, they are printed by the |\NAT@citex|.
%
% The superscript citation prints only the following, not the preceding,
% note, and then only if it is present.
%    \begin{macrocode}
%<subpack>\renewcommand\@cite%
%<!subpack>\newcommand\NAT@cite%
    [3]{\ifNAT@swa\NAT@open\if\relax#2\relax\else#2\ \fi
        #1\if\relax#3\relax\else\NAT@cmt\ #3\fi\NAT@close\else#1\fi}
%<*!subpack>
\newcommand\NAT@citenum[3]{\NAT@open\if\relax#2\relax\else
   #2\ \fi #1\if\relax#3\relax\else\NAT@cmt\ #3\fi\NAT@close}
\newcommand\NAT@citesuper[3]{\unskip\hspace{1pt}\textsuperscript{#1}%
   \if\relax#3\relax\else\ (#3)\fi}
%<!209>\providecommand
%<209>\newcommand
  \textsuperscript[1]{\mbox{$^{\mbox{\scriptsize#1}}$}}
%</!subpack>
%    \end{macrocode}
% \end{macro}\end{macro}\end{macro}\end{macro}
% 
% \begin{macro}{\@citex}
% \changes{5.5}{1995 Mar 21}{Is \cs{let} equal to the numerical or
%      author--year definition}
% \changes{6.1}{1995 Nov 22}{Use \cs{NAT@citeundefined}}
% \begin{macro}{\NAT@citexnum}
% \changes{5.4}{1994 Nov 24}{Make like \LaTeXe{} definition instead of 2.09}
% \changes{5.5}{1995 Mar 21}{Renamed from \cs{@citexnum}}
% \changes{6.0}{1995 Sep 4}{Prints \cs{NAT@num} instead of \cs{b@citeb}}
% \changes{6.4}{1996 Sep 12}{Add \cs{NAT@space} for variable spacing between
%        numericals and superscripts}
% \changes{6.4}{1996 Oct 2}{Add dummy macros for \texttt{hyperref.sty}}
% The original definition of |\@citex| is now used to define |\NAT@citexnum|,
% the version for numerical citations. What it does is to write a
% |\citation| command to the auxiliary file (for \btx), and then parses the
% second argument, the list of citation keys. These keys have to be decoded
% into the actual label text contained in |\b@|\emph{key} (a number or
% author--year), and put together as the argument of |\@cite|. 
% 
% The temporary command |\@citeb| takes on the value of each of the keys in
% turn; |\@citea| is the separator between multiple citations, initially
% set to nothing, becoming |\NAT@sep| plus line-break suppression
% afterwards. The |\@firstofone| macro removes any leading blanks in
% |\@citeb|, making |\cite{key1, key2}| possible. If |\b@|\emph{key} is
% not defined (as on the first run, for example), a warning is printed, and
% a question mark inserted. Finally, the decoded label text is placed into
% an |\hbox| to prevent it from being split.
%  
% As of version 6.0, the numerical label text is to be found in |\NAT@num|,
% whereas previously it was the entire label, |\b@|\emph{key}.
% 
% Note that here, and in later coding, internal names beginning with
% |\@cite..| are employed; these names are used in the original \LaTeX{}
% coding for the same functions. Therefore, if there are other packages
% that have name conflicts with \thestyle, they will also have conflicts with
% standard \LaTeX\ too.
% 
% For compatibility with the \texttt{chapterbib} package of Donald Arseneau,
% add |\@extra@b@citeb| to the cite key; this contains the number of the 
% included file in order to distinguish the same citation in separate files
% which then belong to different bibliography listings.
% \begin{macrocode} 
%<209>\let\@firstofone\@iden
%<!209>\providecommand\@firstofone[1]{#1}
%<*!subpack>
\newcommand\NAT@citexnum{}
\def\NAT@citexnum[#1][#2]#3{\let\@citea\@empty
  \@cite{\@for\@citeb:=#3\do
    {\@citea\def\@citea{\NAT@sep\penalty\@m\NAT@space}%
     \edef\@citeb{\expandafter\@firstofone\@citeb}%
     \if@filesw\immediate\write\@auxout{\string\citation{\@citeb}}\fi
     \@ifundefined{b@\@citeb\@extra@b@citeb}{%
%<209>       {\reset@font\bf ?}\@warning
%<!209>{\reset@font\bfseries?}
%<!209>        \NAT@citeundefined\PackageWarning{natbib}%
       {Citation `\@citeb' on page \thepage \space undefined}}%
     {\NAT@parse{\@citeb}\hbox{\hyper@natlinkstart{\@citeb}%
      \NAT@num\hyper@natlinkend}}}}{#1}{#2}}
%</!subpack>
%    \end{macrocode}
% \end{macro}
%
% \begin{macro}{\NAT@citex}
% \changes{5.2}{1994 Aug 25}{Make like \LaTeXe{} definition instead of 2.09}
% \changes{5.4}{1994 Nov 24}{Change space to command space for text cites}
% \changes{5.4}{1995 Feb 6}{Add test for adjacent citations having equal
%   authors and years}
% \changes{5.5}{1995 Mar 21}{Renamed from \cs{@citex}}
% \changes{6.0}{1995 Sep 21}{Move \cs{ifNAT@full} test from
%          \cs{NAT@parse} to here}
% \changes{6.4}{1996 Oct 2}{Add dummy macros for \texttt{hyperref.sty}}
% The author--year version of |\@citex| is now defined.
% It starts off much the same as the numerical version. However, it wants
% the authors and year for each decoded label text to be separated into
% |\NAT@nm| and |\NAT@date|. This is done by the routine |\NAT@parse|.
% The names of the authors in the previous key are stored in
% |\NAT@last@nm|,
% and if that is the same as the current names, it is not printed again.
% The output citation text is not put into an |\hbox| since line division
% may occur within the author--year citation. The output is formatted
% differently for parenthetical (|\NAT@swa| \meta{true}) and textual
% (|\NAT@swa| \meta{false}) cases.
% 
% |\NAT@nm| will contain either |\NAT@name| or |\NAT@all@names| depending
% on the state of |\ifNAT@full|.
%    \begin{macrocode}
%<subpack>\def\@citex%
%<!subpack>\newcommand\NAT@citex{}
%<!subpack>\def\NAT@citex%
  [#1][#2]#3{\let\@citea\@empty
  \@cite{\let\NAT@nm\@empty\let\NAT@year\@empty
    \@for\@citeb:=#3\do
    {\edef\@citeb{\expandafter\@firstofone\@citeb}%
     \if@filesw\immediate\write\@auxout{\string\citation{\@citeb}}\fi
     \@ifundefined{b@\@citeb\@extra@b@citeb}{\@citea%
%<209>       {\reset@font\bf ?}\@warning
%<!209>       {\reset@font\bfseries ?}\NAT@citeundefined
%<!209>                 \PackageWarning{natbib}%
       {Citation `\@citeb' on page \thepage \space undefined}}%
     {\hyper@natlinkstart{\@citeb}%
     \let\NAT@last@nm=\NAT@nm\let\NAT@last@yr=\NAT@year
     \NAT@parse{\@citeb}\ifNAT@full\let\NAT@nm\NAT@all@names\else
       \let\NAT@nm\NAT@name\fi
%    \end{macrocode}
% Begin parenthetical citation. Check if author names the same as in the
% previous citation, and if so, also check if the years are the same.
% \changes{5.4}{1995 Feb 8}{Add \cs{unskip} to avoid double spacing
%     between years}
% \changes{6.5}{1997 Jan 30}{Add optional notes for textual citations}
% If so, citations are printed `1994a,b'; if a space is wanted between
%  the letters, as `1994a,~b', then give |\NAT@yrsep| as |{,~}|. For this
%  reason, the |\unskip| is needed when the space between years is printed
%  to avoid two spaces.
%    \begin{macrocode}
     \ifNAT@swa
       \ifx\NAT@last@nm\NAT@nm\NAT@yrsep
          \ifx\NAT@last@yr\NAT@year \NAT@exlab\else\unskip\ \NAT@date\fi
       \else\@citea{\NAT@nm}\NAT@aysep
       \ \NAT@date \fi \def\@citea{\NAT@sep\ }%
%    \end{macrocode}
% Begin textual citation. Do same checks for repeated names and years.
% Optional notes are added here, not in |\NAT@cite|, which means for 
% multiple citations, each one will receive the notes in the year part.
% (Pre-notes are questionable for this, but include them anyway.)
%    \begin{macrocode}
     \else
       \ifx\NAT@last@nm\NAT@nm\NAT@yrsep\ifx\NAT@last@yr\NAT@year 
         \NAT@exlab\else \unskip\ \NAT@date\fi
       \else\@citea{\NAT@nm}\ \NAT@open\if\relax#1\relax\else#1\ \fi
         \NAT@date\fi
       \def\@citea{\NAT@close\NAT@sep\ }%
     \fi \hyper@natlinkend
     }}\ifNAT@swa\else\if\relax#2\relax\else\NAT@cmt\ #2\fi
     \NAT@close\fi}{#1}{#2}}
%    \end{macrocode}
%
% \end{macro}\end{macro}
%
% \begin{macro}{\@biblabel}
% \changes{5.5}{1995 Mar 21}{Is \cs{let} equal to the numerical or
%      author--year definition}
% \begin{macro}{\NAT@biblabel}
% \changes{5.5}{1995 Mar 21}{Renamed from \cs{@biblabel}}
% \begin{macro}{\NAT@biblabelnum}
% \changes{5.5}{1995 Mar 21}{Renamed from \cs{@biblabelnum}}
% Similarly for the |\@biblabel| command, that is used to format the
% citation label in the list of references, define the author--year 
% and the numerical versions. For the former, no labels are
% printed in the list of references.
%    \begin{macrocode}
%<subpack>\renewcommand\@biblabel[1]{\hfill}
%<*!subpack>
\newcommand\NAT@biblabel[1]{\hfill}
\newcommand\NAT@biblabelnum[1]{[#1]}
%</!subpack>
%    \end{macrocode}
% \end{macro}\end{macro}\end{macro}
% 
% \begin{macro}{\@bibsetup}
% \changes{5.5}{1995 Mar 21}{Is \cs{let} equal to the numerical or
%      author--year definition}
% \begin{macro}{\NAT@bibsetup}
% \changes{5.5}{1995 Mar 13}{Add \cs{bibhang} for variable hanging
%          indentation}
% \changes{5.5}{1995 Mar 21}{Renamed from \cs{@bibsetup}}
% \begin{macro}{\NAT@bibsetnum}
% \changes{5.1}{1994 Jun 22}{Add \cmd{\if@openbib} as in \LaTeXe}
% \changes{5.5}{1995 Mar 21}{Renamed from \cs{@bibsetnum}}
% \changes{6.1}{1995 Nov 22}{Use \cs{ifNAT@openbib}}
% \changes{6.2}{1996 Mar 05}{Add \cs{bibsep} for linespacing between
%               references}
% \changes{6.5}{1997 Jan 10}{Set \cs{bibsep} to 0 for EGS}
% The macro |\@bibsetup| is called by |\thebibliography| and contains any
% coding for formatting the list of references that may be different for
% numerical and author--year schemes.  For numerical citations, the
% |\labelwidth| must be set to the size of the longest label (the argument
% of |\thebibliography|), and the |\leftmargin| adjusted accordingly. For
% author--year, there are no labels, so |\labelwidth| is zero, all lines
% after the first are indented by 1~em.
% 
% Since |\thebibliography| is not included in the \textsl{AGU} package,
% this macro is left off here too.
%    \begin{macrocode}
%<*!agu>
%<*!egs>
%<*!subpack>
\newcommand\NAT@bibsetnum[1]{\settowidth\labelwidth{\@biblabel{#1}}%
   \setlength{\leftmargin}{\labelwidth}\addtolength{\leftmargin}{\labelsep}%
   \setlength{\itemsep}{\bibsep}\setlength{\parsep}{\z@}%
%<*!209>
   \ifNAT@openbib
     \addtolength{\leftmargin}{\bibindent}%
     \setlength{\itemindent}{-\bibindent}%
     \setlength{\listparindent}{\itemindent}%
     \setlength{\parsep}{0pt}%
   \fi
%</!209>
}
%</!subpack>
%</!egs>
\newlength{\bibhang}
\setlength{\bibhang}{1em}
%    \end{macrocode}
%
% The length |\bibsep| is to be the spacing between listed references.
% This is normally the regular line spacing between |\item|s in a
% \texttt{list} environment, which is |\itemsep| + |\parsep|. These are set
% by |\@listi|. Therefore, in order to set |\bibsep| to the default value,
% call |\@listi| locally and globally set it.
%
%    \begin{macrocode}
\newlength{\bibsep}
%<!egs>{\@listi \global\bibsep\itemsep \global\advance\bibsep by\parsep}
%<egs>\setlength{\bibsep}{\z@}

%<subpack>\newcommand\@bibsetup%
%<!subpack>\newcommand\NAT@bibsetup%
   [1]{\setlength{\leftmargin}{\bibhang}\setlength{\itemindent}{-\leftmargin}%
       \setlength{\itemsep}{\bibsep}\setlength{\parsep}{\z@}}
%</!agu>
%    \end{macrocode}
% \end{macro}\end{macro}\end{macro}
% 
% \begin{macro}{\NAT@set@cites}
% \changes{5.5}{1995 Mar 21}{New macro to set the internal macros to
%         numerical or author--year}
% \changes{6.3}{1996 Jun 17}{Add \cs{natexlab}}
% \changes{6.4}{1996 Sep 12}{Remove \cs{global} from definitions and \cs{let}}
% \changes{6.4}{1996 Sep 12}{Define \cs{NAT@space} differently for 
%        numericals and superscripts}
% The four internal macros that have different definitions depending on the
% citation style have been defined with different names; now set the macros
% to be executed to that set that is actually wanted. This is determined by
% the flags |\ifNAT@numbers| and |\ifNAT@super|. The setting command 
% |\NAT@set@cites| is executed by |\begin{document}|, after which point,
% the citation style should be frozen.
%
% The command |\natexlab| is used in the \texttt{.bst} files to enclose the
% extra labels (the letters added to the dates); these are to be suppressed
% for numerical citations.
%    \begin{macrocode}
%<*!subpack>
\newcommand\NAT@set@cites{\ifNAT@numbers
  \ifNAT@super \let\@cite\NAT@citesuper 
     \let\NAT@space\relax\else
     \let\@cite\NAT@citenum \def\NAT@space{ }\fi
  \let\@citex\NAT@citexnum
  \let\@biblabel\NAT@biblabelnum
  \let\@bibsetup\NAT@bibsetnum
  \def\natexlab##1{}%
 \else
  \let\@cite\NAT@cite
  \let\@citex\NAT@citex
  \let\@biblabel\NAT@biblabel
  \let\@bibsetup\NAT@bibsetup
  \def\natexlab##1{##1}%
 \fi}
%<!209>\AtBeginDocument{\NAT@set@cites}
%</!subpack>
%    \end{macrocode}
% \end{macro}
%
% \changes{6.5}{1997 Jan 30}{Add \texttt{showkeys} coding for \cs{@citex}}
% Since |\citep|, |\citet|, and |\citealt| are all ultimately defined in 
% terms of the |\@citex| command, the best way to get the \texttt{showkeys}
% package to handle these is to let it accommodate only |\@citex|. David
% Carlisle has allowed for \thestyle, but only for |\cite| with variable number
% of arguments, |\citeauthor|, |\citefullauthor|, and |\citeyear|. Add the
% necessary coding at the beginning, after we know if \texttt{showkeys}
% has been loaded or not, and |\@citex| has its final definition.
%    \begin{macrocode}
%<*!209>
\AtBeginDocument{\@ifpackageloaded{showkeys}{%
\let\SK@citex\@citex
       \def\@citex[#1][#2]#3{\SK@\SK@@ref{#3}\SK@citex[#1][#2]{#3}}
}{}}
%</!209>
%    \end{macrocode}
%
% \subsection{The Citations}
%
% \begin{macro}{\citet}
% \changes{6.2}{1996 Jan 11}{Add macro for textual cite in numerical mode}
% \changes{6.5}{1997 Jan 30}{Make this standard textual citation, with notes}
% Textual citations are made with the |\citet| command, which has a starred
% form for full authors, and may have one or two optional arguments for notes.
%
% This command is equivalent to the older syntex |\cite|\marg{key} without
% optional arguments, at least in author--year mode. In numerical mode,
% it prints the authors and then the numerical reference, whereas |\cite|
% will only print the number. This simplifies the switching between the two
% modes with minimal changes in text. One wants to replace ``as shown by Jones
% et~al.\ (1990)'' with ``as shown by Jones et~al.\ [21]''; with |\cite|,
% one would get ``as shown by [21]''. 
%
% This is such a practical idea, that I have decided to make it the standard
% means of textual citing. It now even allows optional notes in textual cites,
% something that I have been asked to provide, and which is impossible under 
% the old syntex with |\cite|.
%
% Note that the flag |\ifNAT@swa| determines if the citation is textual or
% parenthetical, and |\ifNAT@full| whether abbreviated or full author names
% are to be printed.
%    \begin{macrocode}
\newif\ifNAT@full\NAT@fullfalse
\newif\ifNAT@swa
%<209>\renewcommand\citet
%<!209>\DeclareRobustCommand\citet
   {\NAT@swafalse
         \@ifstar{\NAT@fulltrue\NAT@citet}{\NAT@fullfalse\NAT@citet}}
\newcommand\NAT@citet{\@ifnextchar[{\NAT@@citet}{\NAT@@citet[]}}
\newcommand\NAT@@citet{}
\def\NAT@@citet[#1]{\@ifnextchar[{\NAT@@@citet[#1]}{\NAT@@@citet[][#1]}}
\newcommand\NAT@@@citet{}
\def\NAT@@@citet[#1][#2]#3{%
%<!subpack>  \ifNAT@numbers\ifNAT@full\citefullauthor{#3}\else
%<!subpack>  \citeauthor{#3}\fi\ \fi
  \@citex[#1][#2]{#3}}
%    \end{macrocode}
% This command can only handle single keys (in numerical mode).
% \end{macro}
% 
% \begin{macro}{\citep}
% \changes{5.5}{1995 Mar 27}{Add macro as shorthand for \cs{cite[]}}
% \changes{6.5}{1997 Jan 30}{Make this standard parenthetical citation, 
%        with notes}
% Parenthetical citations are made with the |\citep| command, which has a 
% starred
% form for full authors, and may have one or two optional arguments for notes.
%
% This command was originally added by special request to be a simplification
% for |\cite[]|\marg{key}, and I added it grudgingly. After I reprogrammed 
% \thestyle\ to handle author--year and numerical citations with the same
% \texttt{.bst} file, I needed to add the |\citet| command, and this one
% became an obvious companion for parenthetical citations. I have now altered
% its coding to be the standard, being fully equivalent in functionality
% to the older syntax |\cite|\oarg{pre}\oarg{post}\marg{key}.
%    \begin{macrocode}
%<209>\renewcommand\citep
%<!209>\DeclareRobustCommand\citep
   {\NAT@swatrue
         \@ifstar{\NAT@fulltrue\NAT@citep}{\NAT@fullfalse\NAT@citep}}
\newcommand\NAT@citep{\@ifnextchar[{\NAT@@citep}{\NAT@@citep[]}}
\newcommand\NAT@@citep{}
\def\NAT@@citep[#1]{\@ifnextchar [{\@citex[#1]}{\@citex[][#1]}}
%    \end{macrocode}
% \end{macro}
% 
% \begin{macro}{\cite}
% \changes{5.0}{1994 May 18}{Add a second optional argument.}
% \changes{5.3}{1994 Sep 19}{Add starred version to print full author list.}
% \changes{5.4}{1994 Nov 24}{Replace \cmd{\if@tempswa} with \cmd{\ifNAT@swa}.}
% \changes{6.5}{1997 Jan 30}{Declare obsolete, retained for compatibility}
% The |\cite| command was originally used to produce both textual and
% parenthetical citations; with an option argument, even an empty one,
% the citation was parenthical. I now recommend the use of |\citet| and 
% |\citep| instead, retaining |\cite| only for compatibility.
%
% It is equivalent to |\citep| in numerical mode, so it can in fact be 
% used as standard \LaTeX\ syntax with numerical citations.
%    \begin{macrocode}
%<209>\renewcommand\cite
%<!209>\DeclareRobustCommand\cite
    {\@ifstar{\NAT@fulltrue\NAT@cites}{\NAT@fullfalse\NAT@cites}}
\newcommand\NAT@cites{\@ifnextchar [{\NAT@swatrue\NAT@@citep}{\NAT@swafalse
    \NAT@@citep[]}}
% \end{macro}
% 
% \begin{macro}{\citealt}
% \changes{5.5}{1995 Mar 16}{Add macro}
% \changes{6.5}{1997 Jan 30}{Uses \cs{citet} instead of \cs{cite}}
% An alternative form of citation is one in which the parentheses are not
% printed. Since the effect of this command is exactly the same as the
% regular |\citet| but with |\NAT@open| and |\NAT@close| set to blank,
% define it just in that way. The starred version for full authors can be
% retained. Notes are not possible, and would be unnecessary without braces.
% 
% This form has been added at the request of Alexander Holt, who wishes to
% be able to print all three flavours: parenthetical, textual with
% parentheses, and textual without. The |\citealt| command must be able to
% handle multiple references and use the same punctuation scheme as
% |\citet|. 
% 
% By defining |\citealt| in terms of |\citet|, not only do I save
% considerable programming effort, but David Carlisle's \texttt{showkeys}
% package will work with this command too.
%    \begin{macrocode}
%<*!subpack>
%<209>\renewcommand\citealt
%<!209>\DeclareRobustCommand\citealt
    {\@ifstar{\NAT@alts}{\NAT@alt}}
\newcommand\NAT@alt[1]{{\ifNAT@numbers\else
     \renewcommand\NAT@open{}\renewcommand\NAT@close{}\fi\cite{#1}}}
\newcommand\NAT@alts[1]{{\ifNAT@numbers\else
     \renewcommand\NAT@open{}\renewcommand\NAT@close{}\fi\cite*{#1}}}
%</!subpack>
%    \end{macrocode}
% \end{macro}
%
% \begin{macro}{\citeauthor}
% \changes{5.0}{1994 May 18}{Add means to cite authors only.}
% \changes{5.2}{1994 Aug 25}{Enclose \cs{@citenm} in braces.}
% \changes{5.2}{1994 Aug 25}{Add auxiliary file entry.}
% \changes{6.0}{1995 Sep 4}{Check for empty author}
% \changes{6.4}{1996 Oct 2}{Add dummy macros for \texttt{hyperref.sty}}
% \begin{macro}{\citeyear}
% \changes{5.0}{1994 May 18}{Add means to cite year only.}
% \changes{5.2}{1994 Aug 25}{Add auxiliary file entry.}
% \changes{6.0}{1995 Sep 4}{Check for empty year}
% \changes{6.4}{1996 Sep 12}{Extra letter also printed with year number}
% \changes{6.4}{1996 Oct 2}{Add dummy macros for \texttt{hyperref.sty}}
% In an author--year scheme, it is sometimes necessary to refer to the
% authors of a work without the year. It might even be necessary to refer
% to the year alone, but I can hardly imagine that. Both possibilities are
% allowed for here.
%    \begin{macrocode}
\newcommand\citeauthor[1]{\if@filesw\immediate\write
     \@auxout{\string\citation{#1}}\fi
     \@ifundefined{b@#1\@extra@b@citeb}{%
%<209>       {\reset@font\bf ?}\@warning
%<!209>       {\reset@font\bfseries ?}\NAT@citeundefined
%<!209>            \PackageWarning{natbib}%
       {Citation `#1' on page \thepage \space undefined}}%
       {\NAT@parse{#1}\ifx\NAT@noname\NAT@name
%<209>         {\reset@font\bf(author?)}\@warning
%<!209>         {\reset@font\bfseries(author?)}\PackageWarning{natbib}
           {Author undefined for citation `#1' 
%<209>              ^^J
%<!209>             \MessageBreak       
            on page \thepage}\else
         {\hyper@natlinkstart{#1}\NAT@name\hyper@natlinkend}\fi}%
}
\newcommand\citeyear[1]{\if@filesw\immediate\write
     \@auxout{\string\citation{#1}}\fi
     \@ifundefined{b@#1\@extra@b@citeb}{%
%<209>       {\reset@font\bf ?}\@warning
%<!209>       {\reset@font\bfseries ?}\NAT@citeundefined
%<!209>            \PackageWarning{natbib}%
       {Citation `#1' on page \thepage \space undefined}}%
       {\NAT@parse{#1}\if\relax\NAT@date\relax
%<209>         {\reset@font\bf(year?)}\@warning
%<!209>         {\reset@font\bfseries(year?)}\PackageWarning{natbib}
           {Year undefined for citation `#1' 
%<209>              ^^J
%<!209>             \MessageBreak       
            on page \thepage}\else
         {\hyper@natlinkstart{#1}\NAT@date\hyper@natlinkend}\fi}%
}
%    \end{macrocode}
% \end{macro}\end{macro}
% 
% \begin{macro}{\citefullauthor}
% \changes{5.3}{1994 Sep 13}{Add macro for those bib styles with full author
%     list available}
% \changes{6.0}{1995 Sep 4}{Check for empty author}
% \changes{6.4}{1996 Oct 2}{Add dummy macros for \texttt{hyperref.sty}}
% Some of the author--year bibliography styles supported here (\texttt{harvard}
%   and \texttt{newapa}) allow a citation with the full author list, because
%   they have the necessary information in the |\bibitem| entry. With 
%   version~5.3, I change the native form of this entry to allow
%   full authors too.
%
% Since I now have a starred form of |\cite| for the full author list, it
%  might be reasonable to use a starred |\citeauthor| instead of
% |\citefullauthor|. However, that will conflict with the provisions for
% \thestyle{} to be found in Carlisle's \texttt{showkeys} package. That's life.
%    \begin{macrocode}
\newcommand\citefullauthor[1]{\if@filesw\immediate\write
     \@auxout{\string\citation{#1}}\fi
     \@ifundefined{b@#1\@extra@b@citeb}{%
%<209>       {\reset@font\bf ?}\@warning
%<!209>       {\reset@font\bfseries ?}\NAT@citeundefined
%<!209>            \PackageWarning{natbib}%
       {Citation `#1' on page \thepage \space undefined}}%
       {\NAT@parse{#1}\ifx\NAT@noname\NAT@all@names
%<209>         {\reset@font\bf(author?)}\@warning
%<!209>         {\reset@font\bfseries(author?)}\PackageWarning{natbib}
           {Author undefined for citation `#1' 
%<209>              ^^J
%<!209>             \MessageBreak       
            on page \thepage}\else
         {\hyper@natlinkstart{#1}\NAT@all@names\hyper@natlinkend}\fi}%
}
%    \end{macrocode}
% \end{macro}
%
% \subsection{Parsing the Author--Year Entries}
% \begin{macro}{\NAT@parse}
% \changes{5.0}{1994 May 18}{Add parsing command, with additional 
%    font and sub-accent commands relaxed}
% \changes{5.1}{1994 Jun 22}{For \LaTeXe{} of \texttt{1994/06/01}, 
%    can protect font and sub-accents much better}
% \changes{5.2}{1994 Aug 24}{2.09 version to run in compatibility mode}
% \changes{5.3}{1994 Sep 13}{Add full author list}
% \changes{5.3}{1994 Sep 19}{Support starred version of \cs{cite} for full
%    author list}
% \changes{5.3}{1994 Sep 26}{Use \cs{@unexpandable@protect} in place of
%    \cs{noexpand}}
% \changes{5.4}{1995 Feb 6}{Call \cs{@parse@date} to split date into year and
%     extra label}
% \changes{5.5}{1995 May 18}{Rename from \cs{@cite@parse}}
% \changes{6.0}{1995 Sep 4}{Remove \cs{relax} from expanded \cs{@tempa}
%    definition; no longer needed under new system}
% \changes{6.0}{1995 Sep 21}{Prevent expansion of forced space (tilde)}
% The |\bibcite| commands in the auxiliary file associate the numerical and
% author--year information with the |\bibitem| key, which is the argument to
% |\NAT@parse|. As of version~6.0, the information is stored in
% |\b@|\emph{key} for each citation, as four separate items.
% \begin{quote}\begin{tabular}{lll}
%   1. & |\NAT@num| & sequence number or non-author--year label \\
%   2. & |\NAT@date| & year, with possible extra label \\
%   3. & |\NAT@name| & short author list \\
%   4. & |\NAT@all@names| & long author list (optional) 
% \end{tabular}\end{quote}
% In order to define each of the above commands as the appropriate item, it
% is first necessary to expand |\b@|\emph{key} into a temporary command
% |\NAT@temp|, which is then used as the argument of |\NAT@split|, which
% makes the individual assignments.
% 
%    \begin{macrocode}
\newcommand\NAT@parse[1]{{%
%<*209>
      \@ifundefined{documentclass}
       {\let\prm=\relax\let\psf=\relax\let\ptt=\relax\let\pbf=\relax
        \let\psl=\relax\let\psc=\relax\let\pit=\relax\let\pem=\relax
        \let\prmfamily=\relax\let\psffamily=\relax\let\pttfamily=\relax
        \let\pbfseries=\relax\let\pslshape=\relax\let\pscshape=\relax
        \let\pitshape=\relax\let\pmdseries=\relax\let\pupshape=\relax
        \let\pc=\relax \let\pd=\relax \let\pb=\relax}
       {\let\protect=\@unexpandable@protect\let~\relax}%
%</209>
%<!209>     \let\protect=\@unexpandable@protect\let~\relax
     \xdef\NAT@temp{\csname b@#1\@extra@b@citeb\endcsname}}%
     \expandafter\NAT@split\NAT@temp
     \expandafter\NAT@parse@date\NAT@date??????@@%
%<!subpack&!209>     \ifciteindex\NAT@index\fi
}
%    \end{macrocode}
% Some further notes on the above: for 2.09 it is necessary to turn off all the
% protected font commands |\prm| etc.\ when the label text is expanded into
% |\NAT@temp|, for otherwise there will be problems with the New Font
% Selection Scheme (NFSS); the protected sub-accent commands (protected
% versions of |\c|, |\d|, and |\b|) also cause troubles on expansion.
% Admittedly, this is not a good solution, for the list of problem commands
% might be much longer, but for the meantime, this will have to do.
% 
% For \LaTeXe, at least after the official release of
% \texttt{1994/06/01}, commands are made robust in a totally different
% manner, no longer by adding a \texttt{p} in front, but by adding a space
% afterwards. This means the commands in the auxiliary files look like the
% protected version once again, and not raw commands. They are infinitely
% protected. The above problem is easily overcome by making |\protect|
% equal to |\@unexpandable@protect|. A definitive list of commands to 
% protect is no longer necessary. (Initially I used |\noexpand|, but
% that only worked for font commands, not accents. Reading the documentation
% in \texttt{ltdefns.dtx} indicated that this is the better redefinition
% of |\protect|.)
%
% This trick is also applied to the 2.09 version for when it is running
% in the compatibility mode of \LaTeXe. This is needed for the 
% journal-specific coding, more than for the normal package.
% 
% The automatic indexing of citations (|\NAT@index|) is described in
% Section~\ref{sec:idx}.
% 
% \end{macro}
% 
% \begin{macro}{\NAT@split}
% \changes{5.3}{1994 Sep 13}{Changed to support full author list}
% \changes{5.5}{1995 May 14}{Fix coding slightly}
% \changes{5.5}{1995 May 18}{Renamed from \cs{@citez}}
% \changes{6.0}{1995 Sep 4}{Recode for new \cs{bibcite} structure}
% The actual assignments of the author--year and numerical label texts to
% the corresponding commands is carried out by |\NAT@split|. If the long
% author list is missing (argument |#4|), it is set equal to the short
% list.
%    \begin{macrocode}
\newcommand\NAT@split[4]{%
  \gdef\NAT@num{#1}\gdef\NAT@name{#3}\gdef\NAT@date{#2}%
  \gdef\NAT@all@names{#4}%
  \ifx\NAT@noname\NAT@all@names \gdef\NAT@all@names{#3}\fi}
%    \end{macrocode}
% \end{macro}
% 
% \begin{macro}{\NAT@parse@date}
% \changes{5.4}{1995 Feb 6}{Add macro to split date into year and extra label}
% \changes{5.5}{1995 Mar 24}{Put extra label in braces to localize any font
%      changes}
% \changes{5.5}{1995 May 18}{Rename from \cs{@parse@date}}
% With version 5.4, add the macro |\NAT@parse@date| that splits the date up
% into year and extra label, putting them into |\NAT@year| and |\NAT@exlab|.
% This is used to test if authors and years are equal for two adjacent
% citations, in which case, only the extra label is printed. The 
% macro searches the first 4 characters of the date for a letter 
% (|\catcode|=11) which then becomes the extra label. If there are only
% numerals in the first 4 positions (more precisely, non-letters), the 5th
% character is taken as extra label. If there is none, this will be a 
% question mark.
%    \begin{macrocode}
\newcommand\NAT@parse@date{}
\def\NAT@parse@date#1#2#3#4#5#6@@{%
  \ifnum\the\catcode`#1=11\def\NAT@year{}\def\NAT@exlab{#1}\else
  \ifnum\the\catcode`#2=11\def\NAT@year{#1}\def\NAT@exlab{#2}\else
  \ifnum\the\catcode`#3=11\def\NAT@year{#1#2}\def\NAT@exlab{#3}\else
  \ifnum\the\catcode`#4=11\def\NAT@year{#1#2#3}\def\NAT@exlab{#4}\else
    \def\NAT@year{#1#2#3#4}\def\NAT@exlab{{#5}}\fi\fi\fi\fi}
%    \end{macrocode}
% \end{macro}
%
% \subsection{Automatic Index Entries for Citations}\label{sec:idx}
% 
% \begin{macro}{\NAT@idxtxt}
% \changes{6.0}{1995 Sep 21}{Add macro}
% \begin{macro}{\NAT@index}
% \changes{6.0}{1995 Sep 21}{Add macro}
% To make an entry in the index for every citation, I have added
% |\NAT@index| to the |\NAT@parse| command. This will do nothing unless
% |\makeindex| has been issued in the preamble, in which case it writes the
% citation information to the \texttt{.idx} file by means of the standard
% \LaTeX\ internal |\@wrindex|. The actual format of the citation
% information is contained in |\NAT@idxtxt| which a user may redefine if
% needed.
%    \begin{macrocode}
%<*!subpack&!209>
\newcommand\NAT@index{}
\let\NAT@makeindex=\makeindex
\renewcommand\makeindex{\NAT@makeindex
  \renewcommand\NAT@index{\@bsphack\begingroup
     \def~{\string~}\@wrindex{\NAT@idxtxt}}}
\newcommand\NAT@idxtxt{\NAT@name\ \NAT@open\NAT@date\NAT@close}
%    \end{macrocode}
% \end{macro}\end{macro}
% \begin{macro}{\ifciteindex}
% \changes{6.0}{1995 Sep 21}{Add flag}
% The switch for activating automatic citation indexing is defined here.
%    \begin{macrocode}
\newif\ifciteindex \citeindexfalse
%    \end{macrocode}
% \end{macro}
%
% \begin{macro}{\citeindextype}
% \changes{6.0}{1995 Sep 29}{Add macro}
% \begin{macro}{\NAT@index@alt}
% \changes{6.0}{1995 Sep 29}{Add macro}
% The above coding will not work with the package \texttt{index.sty} by
% David~M. Jones. For that package, an alternative definition of |\NAT@index|
% is needed which is activated at the beginning of the document if
% \texttt{index} has been loaded. It thus does not matter in which order
% \thestyle\ or \texttt{index} are loaded.
% 
% With the \texttt{index} package, it is possible to have several index
% files, defined by a |\newindex| command that associates a name with two
% file extensions. By default, the name \texttt{default} is associated with
% \texttt{.idx} and \texttt{.ind}. To make the automatic index entries go to
% a different set, redefine |\citeindextype|.
% 
%    \begin{macrocode}
\newcommand\citeindextype{default}
\newcommand\NAT@index@alt{{\let\protect=\noexpand\let~\relax
  \xdef\NAT@temp{\NAT@idxtxt}}\expandafter\NAT@exp\NAT@temp\@nil}
\newcommand\NAT@exp{}
\def\NAT@exp#1\@nil{\index[\citeindextype]{#1}}

\AtBeginDocument{%
\@ifpackageloaded{index}{\let\NAT@index=\NAT@index@alt}{}}
%</!subpack&!209>
%    \end{macrocode}
% \end{macro}\end{macro}
% 
% \subsection{Writing the Entries to the Auxiliary File}
% 
% \noindent(New to version 6.0.)
%
% The author--year information is transferred from the \btx{} database to
% the document via |\bibitem| commands in the |thebibliography|
% environment. Standard \LaTeX\ simply assigns a running number to these
% entries, and writes that number with the citation key to the auxiliary
% file. For author--year citations, the author list and year must be written
% somehow to the auxiliary file. The form for this is to be
% \begin{quote}
% |\bibcite|\marg{key}|{|\marg{num}\marg{year}\marg{short}\marg{long}|}|
% \end{quote}
% This is the same form as in standard \LaTeX, except that the second
% argument becomes a set of 4 items. Here \emph{num} is the sequential
% number that is to be used if a numerical style is selected, \emph{short}
% is the abbreviate author list (Jones et al.), \emph{long} the full author
% list.
% 
% The author--year information is contained in the optional argument to
% |\bibitem|, but just how it is formatted depends on the system used. The
% native \thestyle\ syntax is
% \begin{quote}
% |\bibitem[|\emph{short}|(|\emph{year}|)|\emph{long}|]|\marg{key}
% \end{quote}
% while the \texttt{chicago} system is
% \begin{quote}
% |\bibitem[\protect\citeauthoryear|\marg{long}\marg{short}\marg{year}|]|%
%   \marg{key}
% \end{quote}
% 
% \begin{macro}{\NAT@ifcmd}
% \changes{6.0}{1995 Sep 4}{Add macro}
% \begin{macro}{\NAT@ifxcmd}
% \changes{6.0}{1995 Sep 4}{Add macro}
% It is necessary to process the optional argument by first testing if it
% contains a command, and if so, to execute that command. Otherwise, insert
% our own command to process the following text.
%    \begin{macrocode}
\newcommand\NAT@ifcmd{\futurelet\NAT@temp\NAT@ifxcmd}
\newcommand\NAT@ifxcmd{\ifx\NAT@temp\relax\else\expandafter\NAT@bare\fi}
%    \end{macrocode}
% This actually only tests if the next token expands to |\relax|, which is
% what |\protect| will be at this point. If so, just procede: the command
% itself follows. Otherwise, insert |\NAT@bare|. (Note, we must also define
% the possible commands like |\citeauthoryear| to do what we want.)
% \end{macro}\end{macro}
% 
% \begin{macro}{\NAT@bare}
% \changes{6.0}{1995 Sep 4}{Add macro}
% The command that is inserted if none is present must parse the native
% syntax for the author--year information. The template is somewhat complex
% because it must be able to test if there are no parentheses at all, in
% which case it activates the alternative, |\NAT@apalk| for the
% \texttt{apalike} syntax. When |\NAT@ifcmd| is invoked, it adds
% |()()\@nil| in order to provide the full template of |\NAT@bare|; if the
% argument itself contains no text in parentheses (not \thestyle{}
% syntax) then |#2| will be empty; otherwise, |#4| catches the extra 
% parentheses.
%    \begin{macrocode}
\def\NAT@bare#1(#2)#3()#4\@nil#5{%
%<*all|apalike|authordate>
  \if\relax#2\relax
  \expandafter\NAT@apalk#1, , \@nil{#5}\else
  \stepcounter{NAT@ctr}%
%</all|apalike|authordate>
  \NAT@wrout{\arabic {NAT@ctr}}{#2}{#1}{#3}{#5}
%<all|apalike|authordate>\fi
}
%    \end{macrocode}
% The |NAT@ctr| above contains the designation of whatever counter is
% selected in the list environment for the \texttt{thebibliography}; it is
% the sequential counter for the entries.
% \end{macro}
%
% \begin{macro}{\NAT@wrout}
% \changes{6.0}{1995 Sep 4}{Add macro}
% \begin{macro}{\NAT@noname}
% \changes{6.0}{1995 Sep 6}{Add macro}
% \changes{6.0}{1995 Sep 21}{Prevent expansion of forced space (tilde)}
% Write the information to the auxiliary file as two arguments to |\bibcite|.
% 
% Note: the 5 arguments are: counter, year, short list, long list, key.
% The two author lists are written enclosed in braces to make sure any
% font commands are kept local. Otherwise, one level of braces disappears for
% authors like |{\em Jones}|.
%
% To test later for the absence of names, define |\NAT@noname| here.
% Define it with |\def|, otherwise it will be `long' and will not test
% correctly with |\NAT@all@names| which is defined with |\gdef|.
%    \begin{macrocode}
\newcommand\NAT@wrout[5]{%
\if@filesw
      {\let\protect\noexpand\let~\relax
       \immediate
       \write\@auxout{\string\bibcite{#5}{{#1}{#2}{{#3}}{{#4}}}}}\fi
\ignorespaces}
\def\NAT@noname{{}}
%    \end{macrocode}
% \end{macro}\end{macro}
% 
% \begin{macro}{\bibitem}
% \changes{6.0}{1995 Sep 4}{Recode from standard}
% \begin{macro}{\@lbibitem}
% \changes{6.0}{1995 Sep 4}{Recode from standard}
% \changes{6.4}{1996 Oct 2}{Add dummy macros for \texttt{hyperref.sty}}
% The above commands are initiated with |\NAT@ifcmd| which decides whether
% |\NAT@bare| or one of the other commands is to be executed on the
% following arguments. It is actually |\@lbibitem| that calls |\NAT@ifcmd|,
% which in turn is invoked by the user command |\bibitem|.
% 
% If |\bibitem| has no optional argument, then we have a numerical citation
% style, which is to be forced by setting |\NAT@stdbst|. Put the sequential
% counter value in as the argument to |\@lbibitem| which is then treated as
% a non-author--year label.
%    \begin{macrocode}
\renewcommand\bibitem{%
  \@ifnextchar[{\@lbibitem}{%
%<!subpack>    \global\NAT@stdbsttrue
    \stepcounter{NAT@ctr}\@lbibitem[\arabic{NAT@ctr}]}}
%    \end{macrocode}
% Write the label with |\item| in the bibliography listing with the current
% value of |\NAT@num|, if it exists. This will either be the sequential
% number or whatever non-author--year text that was given as a special
% `numerical'. Since |\@biblabel| outputs only in numerical mode, this
% label vanishes for author--year mode.
% 
% Finally invoke |\NAT@ifcmd| with the funny template to activate
% |\NAT@bare| or some alternative.
%    \begin{macrocode}
\def\@lbibitem[#1]#2{%
   \@ifundefined{b@#2\@extra@b@citeb}{\def\NAT@num{}}{\NAT@parse{#2}}%
   \item[\hfil\hyper@natanchorstart{#2}\@biblabel{\NAT@num}%
    \hyper@natanchorend]%
    \NAT@ifcmd#1()()\@nil{#2}}
%    \end{macrocode}
% Note: in standard \LaTeX, the sequential list counter is printed directly
% with |\item|, and then written to the auxiliary file so that |\cite| can
% refer to it on the next run. Here, |\item| prints exactly the same thing
% as |\cite| on the current run, if we are in numerical mode, otherwise
% nothing.
%
% \changes{6.3}{1996 Jun 10}{Include \texttt{showkeys} coding if already
%               loaded}
% \changes{6.4}{1996 Jun 27}{Remove from 2.09 coding}
% If the \texttt{showkeys} package has already been loaded, the revised
% version of |\@lbibitem| will not have been saved. In this case, rerun some
% of the coding from that package. (If \texttt{showkeys} is loaded after
% \thestyle, there is no problem.)
%    \begin{macrocode}
%<*!209>
\@ifpackageloaded{showkeys}{%
   \let\SK@lbibitem\@lbibitem
   \def\@lbibitem[#1]#2{%
     \SK@lbibitem[#1]{#2}\SK@\SK@@label{#2}\ignorespaces}}{}
%</!209>
%    \end{macrocode}
% \end{macro}\end{macro}
% 
% \begin{macro}{\ifNAT@stdbst}
% \changes{6.0}{1995 Sep 4}{Add flag}
% If any non-author--year |\bibitem|s have been issued, then force
% numerical mode, since the author--year |\bibitem|s can handle this, but
% the numerical ones cannot work in author--year mode. Do this by inserting
% |\NAT@numberstrue| in the auxiliary file.
%    \begin{macrocode}
%<*!subpack>
\newif\ifNAT@stdbst \NAT@stdbstfalse

%<209>\let\NAT@enddoc=\enddocument
%<209>\def\enddocument
%<!209>\AtEndDocument
  {\ifNAT@stdbst\if@filesw\immediate\write\@auxout{\string
   \global\string\NAT@numberstrue}\fi\fi
%<209>\NAT@enddoc
  }
%</!subpack>
%    \end{macrocode}
% \end{macro}
% 
% \begin{macro}{\bibcite}
% \changes{6.1}{1995 Dec 5}{Add own coding for this macro}
% \changes{6.1a}{1995 Dec 19}{Add own test for changed citations}
% Not wanting to rely on the coding for |\bibcite| in the kernel, I define
% it myself. It is not enough just to `provide' it, for it must do the 
% right thing: write a label named |\b@|\emph{key}. The \LaTeX3 boys
% might decide to keep the command, but have it store the citations with
% a different prefix.
% 
% (I outsmarted myself here again. The test for changed citations and
% references involves reading in the auxiliary file with revised
% definitions for |\label| and |\bibcite|. The December~1995 release does
% this via a generalized labelling command |\@newl@bel|. As a result, I
% must add this test myself. Not so bad, because the warning comes from
% \thestyle\ and explicitly refers to citations instead of just
% `references'.)
%    \begin{macrocode}
%<*!209>
\providecommand\bibcite{}
\renewcommand\bibcite[2]{\@ifundefined{b@#1\@extra@binfo}\relax
     {\NAT@citemultiple
      \PackageWarningNoLine{natbib}{Citation `#1' multiply defined}}%
  \global\@namedef{b@#1\@extra@binfo}{#2}}
\AtEndDocument{\NAT@swatrue\renewcommand\bibcite[2]%
                         {\NAT@testdef{#1}{#2}}}
\newcommand\NAT@testdef[2]{%
  \def\NAT@temp{#2}\expandafter \ifx \csname b@#1\@extra@binfo\endcsname
    \NAT@temp \else \ifNAT@swa \NAT@swafalse
       \PackageWarningNoLine{natbib}{Citation(s) may have
          changed.\MessageBreak
          Rerun to get citations correct}\fi\fi}
%</!209>
%<209>\renewcommand\bibcite[2]{\global\@namedef{b@#1\@extra@binfo}{#2}}
%    \end{macrocode}
% \end{macro}
% 
% \subsection{Other Author--Year Schemes}
% \begin{macro}{\NAT@apalk}
% \changes{4.1}{1993 Oct 4}{Simplify by using comma-space as separator}
% \changes{5.3}{1994 Sep 13}{Full author list equals short list}
% \changes{5.5}{1995 May 18}{Renamed from \cs{@citeapalk}}
% \changes{6.0}{1995 Sep 4}{Recoded for version~6.0}
% If the argument of |\bibitem| contains no command, and if it does not
% conform to the \thestyle{} syntax, then try the \texttt{apalike}
% syntax (\emph{short}\texttt{, }\emph{year}). This has a funny template
% too in order to test for different syntax, which is indicated when |#2|
% is empty. In this case, give up trying to fit it to author--year formats,
% and go over to `numerical' with a special label. Set the flag
% |\NAT@stdbst| to mark this. This forces a numerical citation style.
%    \begin{macrocode}
%<*all|apalike|authordate>
\newcommand\NAT@apalk{}
\def\NAT@apalk#1, #2, #3\@nil#4{\if\relax#2\relax
%<!subpack>  \global\NAT@stdbsttrue
  \NAT@wrout{#1}{}{}{}{#4}\else
  \stepcounter{NAT@ctr}%
  \NAT@wrout{\arabic {NAT@ctr}}{#2}{#1}{}{#4}\fi}
%</all|apalike|authordate>
%    \end{macrocode}
% Note: the |NAT@ctr| is stepped here and everywhere else it is used,
% except when a special `numerical' label is being used. This conforms to
% standard \LaTeX\ which only increments the counter for pure numerical
% labels, without an optional argument to |\bibitem|.
% \end{macro}
% 
% The other schemes of citation labelling may also be accommodated, more
% easily than \texttt{apalike}. It is just necessary to define the particular
% commands, like |\citeauthoryear| to write the arguments to the
% auxiliary file in the required. These commands are detected by
% |\NAT@ifcmd| and simply left to be executed.
% 
% The only complication is that |\citeauthoryear| may be given with two or
% three arguments. This is accommodated by testing the |#3| and changing
% the form of the write command accordingly.
% 
% The form of the template is the same as for |\NAT@bare|, since this
% command is inserted in its place and must therefore conform to its
% syntax.
% 
% \begin{macro}{\citeauthoryear}
% \changes{5.3}{1994 Sep 13}{Allow full authors in citation}
% \changes{6.0}{1995 Sep 4}{Recode for version 6.0}
% For \texttt{newapa.bst} itself, the commands |\citestarts|, |\citeends|,
% as well as |\betweenauthors| must be defined; for others in this group,
% these commands are not needed.
%    \begin{macrocode}
%<*newapa|chicago|all>
\newcommand\citeauthoryear{}
\def\citeauthoryear#1#2#3()()\@nil#4{\stepcounter{NAT@ctr}\if\relax#3\relax
   \NAT@wrout{\arabic {NAT@ctr}}{#2}{#1}{}{#4}\else
   \NAT@wrout{\arabic {NAT@ctr}}{#3}{#2}{#1}{#4}\fi}
\newcommand\citestarts{\NAT@open}
\newcommand\citeends{\NAT@close}
\newcommand\betweenauthors{and}
%</newapa|chicago|all>
%    \end{macrocode}
% \end{macro}
% 
% \begin{macro}{\astroncite}
% \changes{6.0}{1995 Sep 4}{Recode for version 6.0}
% The |\astroncite| command must similarly conform to |\NAT@bare| syntax
% and put its arguments into the auxiliary file in the correct order.
%    \begin{macrocode}
%<*astron|all>
\newcommand\astroncite{}
\def\astroncite#1#2()()\@nil#3{\stepcounter{NAT@ctr}\NAT@wrout{\arabic
{NAT@ctr}}{#2}{#1}{}{#3}}
%</astron|all>
%    \end{macrocode}
% \end{macro}
% 
% \begin{macro}{\citename}
% \changes{6.0}{1995 Sep 4}{Recode for version 6.0}
% The |\citename| command just converts its arguments to the |\NAT@apalk|
% syntax, since its first argument should end in \verb*!, ! as for
% \texttt{apalike}. By making use of |\NAT@apalk| here, it is also possible
% to check for non-conformity of the syntax.
%    \begin{macrocode}
%<*authordate|all>
\newcommand\citename{}
\def\citename#1#2()()\@nil#3{\expandafter\NAT@apalk#1#2, \@nil{#3}}
%</authordate|all>
%    \end{macrocode}
% \end{macro}
% 
% \begin{macro}{\harvarditem}
% \changes{5.3}{1994 Sep 13}{Allow full authors in citation}
% \changes{5.5}{1995 May 18}{Defined in \LaTeXe\ terms}
% The Harvard family is allowed for simply by defining |\harvarditem| in
% terms of |\bibitem| with the \thestyle{} form.
% 
% The |\harvarditem| must allow for the optional argument, which is 
% the short author list. Use the short list in |\bibitem| if it is there,
% otherwise the long list must be taken.
% 
%    \begin{macrocode}
%<*harvard|all>
%<!209>\newcommand\harvarditem[4][]%
%<209>\def\harvarditem{\@ifnextchar[{\@harvarditem}{\@harvarditem[]}}
%<209>\def\@harvarditem[#1]#2#3#4%
    {\if\relax#1\relax\bibitem[#2(#3)]{#4}\else
        \bibitem[#1(#3)#2]{#4}\fi }
%    \end{macrocode}
% The command |\citeasnoun| must be defined since \texttt{harvard.bst}
% can add it for cross-referencing.
% \end{macro}
%
% \begin{macro}{\harvardleft}
% \changes{5.3}{1994 Sep 13}{Add macro}
% \begin{macro}{\harvardright}
% \changes{5.3}{1994 Sep 13}{Add macro}
% \begin{macro}{\harvardyearleft}
% \changes{5.3}{1994 Sep 13}{Add macro}
% \begin{macro}{\harvardyearright}
% \changes{5.3}{1994 Sep 13}{Add macro}
% \begin{macro}{\harvardand}
% \changes{5.3}{1994 Sep 13}{Add macro}
% \begin{macro}{\harvardurl}
% \changes{5.3}{1994 Sep 13}{Add macro}
% The \texttt{harvard} package has been updated for \LaTeXe, and includes
% some new punctuation commands in the accompanying \texttt{bst} files.
% (Maybe they were always there and I overlooked them.) In order for
% \thestyle{} to work with these, it needs to define them. The |left| and 
% |right| commands have equivalents already; |\harvardurl| is defined
% as in \texttt{harvard.sty}. It is necessary to define |\harvardand|
% conditionally, and only later, in case it has already been set to
% something else by a |\bibstyle@xxx| (e.g., \texttt{agsm} sets it
% to |\&|.)
%    \begin{macrocode}
\newcommand\harvardleft{\NAT@open}
\newcommand\harvardright{\NAT@close}
\newcommand\harvardyearleft{\NAT@open}
\newcommand\harvardyearright{\NAT@close}
%<!209>\AtBeginDocument{\providecommand{\harvardand}{and}}
%<209>\def\harvardand{and}
\newcommand\harvardurl[1]{\textbf{URL:} \textit{##1}}
%</harvard|all>
%    \end{macrocode}
% \end{macro}\end{macro}\end{macro}\end{macro}\end{macro}\end{macro}
%
% \subsection{The Bibliography Listing}
% Changes must be made to the \texttt{thebibliography} environment for
% author--year citations. This is done be defining the environment with the
% features common to both schemes, and then by invoking |\@bibsetup|, which
% is either the author--year or numerical version.
% 
% \begin{macro}{\bibsection}
% \changes{4.2}{1993 Oct 22}{Make specific for JGR, GRL, NLINPROC}
% \changes{5.5}{1995 May 18}{Use \cs{renewcommand}}
% \changes{6.2}{1996 Apr 15}{Modified for \texttt{amsart} and \texttt{amsbook}}
% \changes{6.4}{1996 Sep 1}{Modified for \texttt{sectionbib} option}
% \changes{6.5}{1997 Feb 5}{Allow for \cs{bib@headings} in KOMA scripts}
% First, |\bibsection| is defined according to the main style: if chapters
% exist, then the bibliography is a numberless chapter, otherwise a
% numberless section. For the specific journals, |\bibsection| takes on
% special definitions.
% 
% Since the AMS classes \texttt{amsart} and \texttt{amsbook} redefine
% |\section*| in a way that is incompatible with the way it is used in 
% the \texttt{bibliographystyle} environment, it is necessary to remove the
% |\@mkboth| from the section title. Since these classes do not use this
% command at this point, there is no loss of functionality.
%
% The packages of the KOMA scripts make use of |\bib@heading| which functions
% much like |\bibsection|. If it is defined, let |\bibsection| be equal to it.
%    \begin{macrocode}
%<*!agu&!egs>
%<!209>\providecommand\bibsection{}
%<209>\def\bibsection{}
\@ifundefined{chapter}%
  {\renewcommand\bibsection{\section*{\refname
    \@mkboth{\MakeUppercase{\refname}}{\MakeUppercase{\refname}}}}}
  {\@ifundefined{NAT@sectionbib}%
    {\renewcommand\bibsection{\chapter*{\bibname
     \@mkboth{\MakeUppercase{\bibname}}{\MakeUppercase{\bibname}}}}}
    {\renewcommand\bibsection{\section*{\bibname
     \ifx\@mkboth\@gobbletwo\else\markright{\MakeUppercase{\bibname}}\fi}}}}
%<*!subpack>
%<*!209>
\@ifclassloaded{amsart}%
  {\renewcommand\bibsection{\section*{\refname}}{}}{}
\@ifclassloaded{amsbook}%
  {\renewcommand\bibsection{\section*{\bibname}}{}}{}
%</!209>
\@ifundefined{bib@heading}{}{\let\bibsection\bib@heading}
%</!subpack>
%</!agu&!egs>
%<egs>\def\bibsection{\if@draft\newpage\fi
%<egs>    \noappendix\section*{\refname}}
%    \end{macrocode}
% \end{macro}
% 
% \begin{environment}{thebibliography}
% \changes{4.1b}{1993 Oct 18}{Add \cmd{\bibfont}, defaulted
%                to \cmd{\relax}}
% \changes{4.2}{1993 Oct 22}{Add coding for JGR, GRL, NLINPROC}
% \changes{4.2}{1993 Nov 20}{Remove coding for AGU, since AGU-supplied
%         coding is adequate but different}
% \changes{5.1}{1994 Jun 22}{Put \cmd{\bibfont} at start so \cmd{\list}
%     uses list parameters appropriate for the font size}
% \changes{5.1}{1994 Jun 22}{Add \cmd{\if@openbib} as in \LaTeXe}
% \changes{5.5}{1995 Mar 27}{Add special cross-ref citations for compatibility
%     with \texttt{harvard}, \texttt{chicago}, and \texttt{named}}
% \changes{6.1}{1995 Nov 22}{Replace \cs{if@openbib} with \cs{ifNAT@openbib}}
% \changes{6.1}{1995 Dec 4}{Replace counter \texttt{enumiv} with
%    \texttt{NAT@ctr} and refer to it explicitly, not with \cs{@listctr}}
% \changes{6.1}{1995 Dec 4}{Use \cs{PackageWarning} for empty
%      bibliography}
%
% The \texttt{thebibliography} environment is defined much as normal, except
% that |\bibsetup| contains the special features for numerical or
% author--year styles. The command |\bibfont| permits permits different
% font sizes or styles to be used in the list. For example, for 
% \textsl{Nonlinear Processes} it is |\small|. This means |\bibfont| must
% be issued before the |\list| command, since the list parameters depend
% on the current font size. For \textsl{AGU}, the whole
% |\thebibliography| is left off, since the AGU-supplied coding will
% do. It is only necessary to add |\noappendix| to the AGU coding.
% 
% I have replaced |\if@openbib| with |\ifNAT@openbib| in order to be
% independent of the standard classes; this flag disappeared with the
% December~1995 release of \LaTeX, so I do it myself just to be sure.
% 
% At the same time,
% I have also decided to use my own counter |NAT@ctr| for the numericals;
% previously I used |enumiv| as in the standard, and refered to it
% elsewhere with |\@listctr|, but now I refer to it explicitly. This is to
% avoid trouble if |\@listctr| should ever disappear. (The value of
% |\@listctr| is set by |\usecounter|.)
%    \begin{macrocode}
\newcounter{NAT@ctr}
%<*!agu>
\renewenvironment{thebibliography}[1]{%
 \bibfont\bibsection\parindent \z@\list
   {\@biblabel{\arabic{NAT@ctr}}}{\@bibsetup{#1}%
    \setcounter{NAT@ctr}{0}}%
%<*!209&!subpack>
    \ifNAT@openbib
      \renewcommand\newblock{\par}
    \else
      \renewcommand\newblock{\hskip .11em \@plus.33em \@minus.07em}%
    \fi
%</!209&!subpack>
%<209|subpack>      \def\newblock{\hskip .11em plus.33em minus.07em}%
    \sloppy\clubpenalty4000\widowpenalty4000
    \sfcode`\.=1000\relax
%    \end{macrocode}
% The files \texttt{chicago.bst} and \texttt{named.bst} can add the
% commands |\citeN| and |\shortcite| respectively to the \texttt{.bbl}
% files, for cross-referencing. Similarly, \texttt{harvard.bst} can add
% |\citeasnoun|. These commands must therefore be accommodated, 
% as least within the |thebibliography| environment.
%    \begin{macrocode}
%<newapa|chicago|all>    \let\citeN\cite \let\shortcite\cite
%<harvard|all>    \let\citeasnoun\cite
 }{\def\@noitemerr{%
%<209>  \@warning
%<!209>  \PackageWarning{natbib}
     {Empty `thebibliography' environment}}%
  \endlist\vskip-\lastskip}
%<!egs>\let\bibfont=\relax
%<egs>\let\bibfont=\small
%</!agu>
%<*agu>
\let\aguthebib=\thebibliography
\def\thebibliography#1{\noappendix\aguthebib{#1}}
%</agu>
%    \end{macrocode}
% \end{environment}
% 
% \subsection{Compatibility Commands}
% \begin{macro}{\reset@font}
% \begin{macro}{\refname}
% \changes{6.4}{1996 Jun 18}{Provide name with \LaTeXe\ too}
% \begin{macro}{\bibname}
% \changes{6.4}{1996 Jun 18}{Provide name with \LaTeXe\ too}
% For older implementations of \LaTeX~2.09 (before December, 1991) the
% |\reset@font| command does not exist. It is defined to be |\relax| under
% normal \LaTeX{}, but does more under NFSS. In case it is not defined, add
% it here. The naming commands |\bibname| and |\refname| are similarly
% unknown for this old \LaTeX. They are sometimes even unknown in non-standard
% classes that are used in place of \texttt{article}, \texttt{report}, and
% \texttt{book}.
%    \begin{macrocode}
%<*!209>
\providecommand\reset@font{\relax}
\providecommand\bibname{Bibliography}
\providecommand\refname{References}
%</!209>
%<*209>
\@ifundefined{reset@font}{\let\reset@font=\relax}{}
%<!agu&!egs>\@ifundefined{bibname}{\newcommand\bibname{Bibliography}}{}
\@ifundefined{refname}{\newcommand\refname{References}}{}
%</209>
%    \end{macrocode}
% \end{macro}\end{macro}\end{macro}
%
% \begin{macro}{\NAT@citeundefined}
% \changes{6.1}{1995 Nov 22}{Add macro}
% The December~1995 release of \LaTeX\ considered renaming the command
% |\G@refundefinedtrue| for flagging undefined references. However, since a
% number of packages, including \thestyle, were affected, it was decided to
% retain the old name for the new coding. I had already decided to go for
% an independent coding, which does exactly the same thing.
%    \begin{macrocode}
%<*!209>
\newcommand\NAT@citeundefined{\gdef \NAT@undefined {%
    \PackageWarningNoLine{natbib}{There were undefined citations}}}
\let \NAT@undefined \relax
%    \end{macrocode}
% \end{macro}
% \begin{macro}{\NAT@citemultiple}
% \changes{6.1}{1995 Dec 6}{Add macro}
% Add the warning about multiply-defined citations, as part of the
% do-it-yourself |\bibcite| definition.
%    \begin{macrocode}
\newcommand\NAT@citemultiple{\gdef \NAT@multiple {%
    \PackageWarningNoLine{natbib}{There were multiply defined citations}}}
\let \NAT@multiple \relax
\AtEndDocument{\NAT@undefined\NAT@multiple}
%</!209>
%    \end{macrocode}
% \end{macro}
%
% \begin{macro}{\@mkboth}
% \changes{6.1}{1995 Dec 4}{Provide macro in case it is removed from
%     kernel}
% Having been burned by the December~1995 update on several internal
% commands, I have decided to make \thestyle\ as independent as
% possible, or at least to avoid it crashing when internals are changed or
% disappear in future. The |\@mkboth| is used in the bibliography listing,
% and is the same as |\makeboth| with \texttt{headings} pagestyle. Provide
% a dummy in case it disappears. If the kernel should use a different
% command in future, the headings will not function, but at least there
% will be no crash. (Also needed for very old versions of \LaTeX~2.09.)
%    \begin{macrocode}
%<209>\@ifundefined{@mkboth}{\newcommand\@markboth[2]{}}{}
%<!209>\providecommand\@mkboth[2]{}
%    \end{macrocode}
% \end{macro}
% 
% \begin{macro}{\MakeUppercase}
% \changes{6.1}{1995 Dec 4}{Provide compatibility for older versions}
% The |\MakeUppercase| command replaces |\uppercase| in \LaTeXe\ of
% June~1995. Provide it for earlier versions.
%    \begin{macrocode}
%<209>\let\MakeUppercase\uppercase
%<!209>\providecommand\MakeUppercase{\uppercase}
%    \end{macrocode}
% \end{macro}
%
% \begin{macro}{\@extra@b@citeb}
% \changes{6.4}{1996 Sep 1}{Add macro for \texttt{chapterbib.sty}}
% \begin{macro}{\@extra@binfo}
% \changes{6.4}{1996 Sep 1}{Add macro for \texttt{chapterbib.sty}}
% Two commands used by the \texttt{chapterbib} package, but which must 
% be defined to be nothing for use without that package. Do not |\let| them
% be |\relax|, for that package checks if they exist (i.e., not |\relax|)
% and redefines things like |\@citex|, which we do not want to have happen.
% The necessary redefinitions are already included in \thestyle.
%    \begin{macrocode}
%<209>\@ifundefined{@extra@b@citeb}{\def\@extra@b@citeb{}}{}
%<!209>\providecommand{\@extra@b@citeb}{}
\gdef\@extra@binfo{}
%    \end{macrocode}
% \end{macro}\end{macro}
%
% \begin{macro}{\hyper@natanchorstart}
% \changes{6.4}{1996 Oct 2}{Provide macro for \texttt{hyperref} package}
% \begin{macro}{\hyper@natanchorend}
% \changes{6.4}{1996 Oct 2}{Provide macro for \texttt{hyperref} package}
% \begin{macro}{\hyper@natlinkstart}
% \changes{6.4}{1996 Oct 2}{Provide macro for \texttt{hyperref} package}
% \begin{macro}{\hyper@natlinkend}
% \changes{6.4}{1996 Oct 2}{Provide macro for \texttt{hyperref} package}
% The \texttt{hyperref} package of Sebastian Rahtz and Yannis Haralambous 
% defines these macros to
% make hypertext links and references. Without that package, these macros
% must be turned off. The order of loading \thestyle\ and \texttt{hyperref} 
% should not be important.
%    \begin{macrocode}
%<*!209>
\providecommand\hyper@natanchorstart[1]{}
\providecommand\hyper@natanchorend{}
\providecommand\hyper@natlinkstart[1]{}
\providecommand\hyper@natlinkend{}
%</!209>
%<*209>
\@ifundefined{hyper@natanchorstart}{\def\hyper@natanchorstart#1{}}{}
\@ifundefined{hyper@natanchorend}{\def\hyper@natanchorend{}}{}
\@ifundefined{hyper@natlinkstart}{\def\hyper@natlinkstart#1{}}{}
\@ifundefined{hyper@natlinkend}{\def\hyper@natlinkend{}}{}
%</209>
%    \end{macrocode}
% \end{macro}\end{macro}\end{macro}\end{macro}
% 
% \subsection{Sorting and Compression of Numbers}
% The \texttt{cite} and (obsolete) \texttt{citesort} packages by Donald
% Arseneau provide the coding for printing multiple citation numbers in
% ascending order, and with a range where applicable. That is, if the
% normal result of |\cite{..}| with several keys were [5, 3, 8, 4], the new
% result would be [3--5, 8].
% 
% Since \texttt{cite.sty} totally reimplements the citation commands (as
% does \thestyle) it is not possible to make the two compatible. Instead, I
% have removed the necessary coding from \texttt{cite.sty}, converted it to
% \thestyle\ format, and added it here. 
% 
% To activate the sorting and compression of numerical citations, load
% \thestyle\ with the options \texttt{numbers,sort}.
%  This coding has no effect on author--year citations.
%
% \begin{macro}{\NAT@citen}
% \changes{6.4}{1996 Oct 11}{Add coding from \texttt{cite.sty}}
% It is not included in the \LaTeX~2.09 version, nor in the journal packages
% since it is turned on by a package option, something missing in both of
% them. Besides, the journals are author--year mode.
%    \begin{macrocode}
%<*!subpack&!209>
\ifnum\NAT@sort=1
\newcommand\NAT@citen[1]{%
\nocite{#1}%
\@tempcntb\m@ne
\let\@h@ld\relax
\let\@citea\@empty
\let\@celt\delimiter 
\def\@cite@list{}% 
\@for \@citeb:=#1\do{\@make@cite@list}%
\@tempcnta\m@ne
\let\@celt\@compress@cite \@cite@list 
\@h@ld}

\begingroup \catcode`\_=8 
\gdef\@make@cite@list{%
     \edef\@citeb{\expandafter\@firstofone\@citeb}%
     \@ifundefined{b@\@citeb\@extra@b@citeb}{%
      \@citea{\reset@font\bfseries?}\def\@citea{\NAT@sep\penalty\@m\NAT@space}%
      \NAT@citeundefined\PackageWarning{natcite}%
      {Citation `\@citeb' on page \thepage\space undefined}}%
    {\NAT@parse{\@citeb}%
      \ifcat _\ifnum\z@<0\NAT@num _\else A\fi % a positive number, put in list
       \@tempcnta\NAT@num \relax
       \ifnum \@tempcnta>\@tempcntb % new highest, add to end (efficiently)
          \edef\@cite@list{\@cite@list \@celt{\NAT@num}}%
          \@tempcntb\@tempcnta
       \else % arbitrary number: insert appropriately
          \edef\@cite@list{\expandafter\@sort@celt \@cite@list \@gobble @}%
       \fi
    \else % citation is not a number, output immediately
       \@citea \NAT@num \def\@citea{\NAT@sep\penalty\@m\NAT@space}%
 \fi}}
\endgroup

\def\@compress@cite#1{%
  \advance\@tempcnta\@ne
  \ifnum #1=\@tempcnta  
     \ifx\@h@ld\relax   
        \edef\@h@ld{\@citea #1}%
     \else              
        \def\@h@ld{\hbox{--}\penalty\@m #1}%
     \fi 
  \else  
     \@h@ld \@citea #1\let\@h@ld\relax
  \fi \@tempcnta#1\def\@citea{\NAT@sep\penalty\@m\NAT@space}%
}

\def\@sort@celt#1#2{\ifx \@celt #1%
     \ifnum #2<\@tempcnta 
        \@celt{#2}%
        \expandafter\expandafter\expandafter\@sort@celt 
     \else
        \@celt{\number\@tempcnta}\@celt{#2}% 
  \fi\fi}
\def\NAT@citexnum[#1][#2]#3{\let\@citea\@empty
  \@cite{\NAT@citen{#3}}{#1}{#2}}
\fi
%</!subpack&!209>
%    \end{macrocode}
% \end{macro}
%
% \subsection{Local Configuration}
% \changes{5.2}{1994 Aug 25}{Add reading in of \texttt{natbib.cfg}}
% \changes{5.5}{1995 Mar 21}{Local configuration read in at end}
% To accommodate local additional |\bibstyle@| definitions, 
% or any other modifications to the coding, read in
% configuration file \texttt{natbib.cfg} if it exists. 
%    \begin{macrocode}
%<*!subpack&!209>
\InputIfFileExists{natbib.cfg}
       {\typeout{Local config file natbib.cfg used}}{}
%</!subpack&!209>
%</package>
%    \end{macrocode}
% 
% \Finale