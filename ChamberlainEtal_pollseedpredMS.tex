\documentclass[12pt]{article}
\usepackage{natbib}
\usepackage{textcomp}
\usepackage[left]{lineno}
\linenumbers
 
\title{Biotic simplification of plant mutualists and antagonists in agricultural landscapes}
\author{Scott Chamberlain \\ Jennifer Rudgers \\ Ken Whitney}

\begin{document}
\maketitle

\begin{center}
Department of Ecology and Evolutionary Biology, Rice University, Houston, TX USA
\end{center}

\begin{center}
For submission to Oecologia as an Original Research Paper
\end{center}

\begin{center}
\date{Submitted 1 February 2012}
\end{center}
\newpage

\section{Abstract}

write abstract here

\section{Introduction}
Anthropogenic disturbances, such as urbanization, fragmentation, and introduction of invasive species, can homogenize biotic communities by reducing the variation in species composition across locations. For example, urbanization has caused the homogenization of both plant and bird communities relative to natural areas \citet{McKinney2006}. Global agricultural intensification has also produced biotic homogenization. For example, in Europe, increased pesticide use has led to increased similarities in both bee and hemipteran communities relative to non-agricultural areas \citet{Dormann2007}. Croplands, pastures, and rangelands constituted ~50 percent of the global vegetated land surface as of 2005 \citep{Foley2005}. Thus, agriculture has the potential to significantly impact diversity of natural communities. 

Given the fact that biotic communities are being altered in agricultural landscapes relative to pristine landscapes, what framework can be used to think about the mechanisms and consequences of this change?  \citet{Rand2006} provided a framework based loosely around spatial flows \citep[cf][]{Polis1997} for understanding changes in predator communities in agricultural landscapes.  Here, we propose a mutualist-antagonist (MA) framework to simultaneously understand the impacts of changes in biotic communities on crop and wild plants (Chamberlain and Burke, unpublished; Fig 1.). The MA framework predicts that abundance and community composition of mutualists and antagonists should be altered in different ways in agricultural landscapes.  For example, farmers often manage crop plant antagonists by spraying pesticides and tilling soil, thereby altering abundance of plant antagonists near crops.  Whereas, farmers either do not purposefully alter plant mutualists, but do sometimes supplement mutualists in crop fields. 

What is the utility of an MA framework?  Simultaneous selection on the same trait, or (ecological pleiotropy), should be common for traits that both plant mutualists and antagonists use as cues \citep{Strauss2004}. For example, selection on floral traits is likely to exhibit greater spatial variability if there are conflicting pressures from pollinators versus seed predators \citep[e.g., ][]{Cariveau2004}. In addition, conflicting selection pressures from mutualists and antagonists enhance phenotypic variation in natural populations relative to selection mediated by only one interaction type \citep{Irwin2003,Siepielski2010}.

Here, we BLA BLA BLA. We ask the following two specific questions: 1) Do mutualists and antagonists differ in abundance near vs. far from sunflower crops?; and 2) Do mutualists and antagonists differ in community structure/diversity near vs. far from sunflower crops?

\section{Methods}
\subsection{Study system}
Cultivated \textit{Helianthus annuus} and its native congeners (sunflowers; Asteraceae) provide a highly tractable system for studying how agriculture alters the evolutionary trajectories of native species.  First, native \textit{Helianthus} commonly occur along the borders of sunflower crop fields \citep{Burke2002}.  Second, in sunflower growing regions in the US, crop and wild sunflowers can overlap for 5-6 mo. in flowering phenology (K. Whitney, pers. obs.), leading to high potential for shared pollinators (mutualists) and seed predators (antagonists) among crop and wild sunflowers. Texas hosts 20 native \textit{Helianthus} species, many of which produce viable, hybrid offspring with crop sunflowers \citep{Whitton1997,Linder1998}, a further indication of shared insect pollinators. Third, as Asteraceae have sporophytic self-incompatibility \citep{Linder1998}, self pollen grains do not germinate pollen tubes allowing for the quantification of outcross pollen grains deposited by pollinators. Finally, my target native species, \textit{Helianthus annuus texanus}, is an annual, which is ideal for measuring lifetime fitness and selection in nature. 

A diverse biotic community interacts with native and crop sunflowers. In general, the pollinator communities of both crop and wild sunflowers are dominated by several hundred species of bees Hurd1980, with honey bees particularly prevalent in crop sunflowers \cite{Greenleaf2006}. Seed predators attack both native and crop sunflowers, and their species-specific damage to sunflower seeds is easily quantified \citep{Whitney2006}. These biotic communities influence selection on sunflower traits \citep{Whitney2006}.

\subsection{Study sites and design}
In an observational study in 2009, at three naturally occurring populations of \textit{H.a. texanus} near to (\textless 10 m), and four populations far from ($\sim$ 2.5 km) crop sunflowers, I marked with metal tags approximately 100 plants per population (Fig 1).  

In experimental studies in 2010, we manipulated the proximity of H. a. texanus to crop sunflowers: Near [array of \textit{H.a. texanus} \textless 10 m from the crop] vs. far [array $\sim$ 2.5 km from any sunflower crop, and near natural habitat].  Plots were replicated at each of five farms in TX (Fig 1.).  The proximity treatment was crossed factorially with a seed origin treatment (seeds from two wild populations collected in 2009; indicated in Fig. 1) to enhance the generality of results.  
	
In 2011, we used the exact same design as 2010, but only used two of the five sites used in 2010 (Marek, Beakley).

\subsection{Pollinators}
We sampled pollinators using two methods: direct observations on our study sunflowers, and water bowl traps.  In direct observations we observed up to 30 plants in each plot for 2-5 min per plant, over 4-6 observation periods during the flowering period (May-September).  A pollinator visit was recorded when we observed a visitor making contact with anthers, stigmas, or both.  Pollinators that could not be identified in the field were collected.  

The water bowl trap method is the most common way to estimate pollinator abundance, is the most efficient method to capture as much diversity as possible, and is the least prone to observer bias \citep{Westphal2008}.  This method catches the subset of pollinators that visit \textit{H.a. texanus}, as well as pollinators that do not visit this plant species; although, there are few other resources for pollinators besides crop sunflowers or the \textit{H.a. texanus} plants in our plots.  Thus, the direct observations are more relevant from the plants perspective.  We set out three to six bowls, of three different colors (white, blue, yellow), at each of two dates throughout the flowering period each year.  We filled bowls with water and few drops of soap to break surface tension.  Bowls were collected after 24 hrs, samples placed in 70\% ethanol, and sorted to the lowest possible taxonomic level following \citet{Michener2000}.    

\subsection{Seed predators}
We quantified abundance of seed predators on all plants in each sunflower plot by putting net bags on three to six inflorescences per plant after pollination, but before shattering (seed drop) occurred, to allow ample time for seed predators to interact with the inflorescence.  We collected these bagged heads at the end of the season, after seeds in inflorescences had matured, and plants had senesced. We pooled all inflorescences, and then sub-sampled $\sim$ 80 seeds with x10 dissecting microscope to quantify species-specific damage.  One damaged seed is assumed equivalent to one individual seed predator; so we take damaged seeds as equivalent to number of individual seed predators. 

\subsection{Data analysis}
\subsubsection{Abundance}  
We compared abundance of pollinators and seed predators separately using either ANOVAs (normality assumption met), or GLMs (normality not met).  In all these models, we included as explanatory factors site, seed source, and proximity (near of far), and their interactions.  We followed significant terms in the model with Tukey post-hoc tests (family-wise $\alpha$ = 0.05).  Seed predators were modeled as XXX with a XXX error distribution.  

\subsubsection{Richness and evenness}
We calculated richness by XXXXX.  We calculated evenness using XXXX.  To determine if differences in richness were due to differences in abundance between treatment levels, we generated sample-based rarefaction curves using each plot as a replicate with 1000 randomization runs using the vegan package \citep{Oksanen2010}.  We used Chao1 as the estimator of richness \citep{Chao1984}.  All analyses were done in R v.2.14 \citep*{Team2008}.

\subsubsection{Community structure}
We conducted non-linear multidimensional scaling analyses to collapse down pollinator species into a few axes to better visualize potential differences in pollinator communities among factors.  We DETAILS OF NMS.

Furthermore, we tested for differences in pollinator community structure in a few different ways.  First, we 

\section{Results}
\subsection{Do mutualists and antagonists differ in abundance near vs. far from sunflower crops?}

\subsection{Do mutualists and antagonists differ in community structure/diversity near vs. far from sunflower crops?}


\section{Discussion}

\section{Conclusion}

\section{Acknowledgements}


\newpage 
\bibliographystyle{ecology_letters2}
\bibliography{/Users/ScottMac/github/SChamberlain/work/pollseedpred} 

\newpage 
\textbf{Figure captions}
Fig. 1.  Diagram of the mutualist-antagonist framework for understanding evolutionary change for native plants in agricultural landscapes.  

Fig. 2.  Map of study sites in 2009, 2010, and 2011.  

Fig. 3.  Rarefaction curves for (a) pollinators and (b) seed predators. 

Fig. 4.  Abundance across three years for (a-c) pollinators and (d-f) seed predators. 

Fig. 5.  Non-metric multidimensional scaling ordination plots across three years for (a-c) pollinators and (d-f) seed predators. 

\end{document}
