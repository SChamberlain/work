\documentclass[12pt]{article}

\title{BIOTIC SIMPLIFICATION OF PLANT MUTUALISTS AND ANTAGONISTS IN AGRICULTURAL LANDSCAPES}
\author{Scott Chamberlain \\ Jennifer Rudgers \\ Ken Whitney}
\date{XXXX}

\begin{document}
\maketitle

\section{Introduction}
Anthropogenic disturbances, such as urbanization, fragmentation, and introduction of invasive species, can homogenize biotic communities by reducing the variation in species composition across locations. For example, urbanization has caused the homogenization of both plant and bird communities relative to natural areas (Mckinney 2006). Global agricultural intensification has also produced biotic homogenization. For example, in Europe, �increased pesticide use has led to increased similarities in both bee and hemipteran communities relative to non-agricultural areas (Dormann et al. 2007). Croplands, pastures, and rangelands constituted ~50 percent of the global vegetated land surface as of 2005 (Foley et al. 2005). Thus, agriculture has the potential to significantly impact diversity of natural communities. 

It is important to tease apart the selective effects mediated by plant mutualists versus plant antagonists, as their abundance and community composition may be differentially affected by agriculture. Simultaneous selection on the same trait, or �ecological pleiotropy,� should be common for traits that both plant mutualists and antagonists use as cues (Strauss \& Irwin, 2004). For example, selection on floral traits is likely to exhibit greater spatial variability if there are conflicting pressures from pollinators versus seed predators (e.g., Cariveau et al., 2004). In addition, conflicting selection pressures from mutualists and antagonists enhance phenotypic variation in natural populations relative to selection mediated by only one interaction type (Irwin et al., 2003; Siepielski \& Benkman, 2010).

Do mutualists and antagonists differ in abundance near vs. far from sunflower crops?  Do mutualists and antagonists differ in community structure/diversity near vs. far from sunflower crops?

\section{Methods}
\subsection{Study system}
Cultivated Helianthus annuus and its native congeners (sunflowers; Asteraceae) provide a highly tractable system for studying how agriculture alters the evolutionary trajectories of native species.  First, native Helianthus commonly occur along the borders of sunflower crop fields (Burke et al. 2002).  Second, in sunflower growing regions in the US, crop and wild sunflowers can overlap for 5-6 mo. in flowering phenology (K. Whitney, pers. obs.), leading to high potential for shared pollinators (mutualists) and seed predators (antagonists) among crop and wild sunflowers. Texas hosts 20 native Helianthus species, many of which produce viable, hybrid offspring with crop sunflowers (Whitton et al. 1997; Linder et al. 1998), a further indication of shared insect pollinators. Third, as Asteraceae have sporophytic self-incompatibility (Linder et al. 1998), self pollen grains do not germinate pollen tubes, allowing for the quantification of outcross pollen grains deposited by pollinators. Finally, my target native species, Helianthus annuus texanus, is an annual, which is ideal for measuring lifetime fitness and selection in nature. 

A diverse biotic community interacts with native and crop sunflowers. In general, the pollinator communities of both crop and wild sunflowers are dominated by several hundred species of bees (Hurd  Jr. et al. 1980), with honey bees particularly prevalent in crop sunflowers (Greenleaf and Kremen 2006). Seed predators attack both native and crop sunflowers, and their species-specific damage to sunflower seeds is easily quantified (Whitney et al. 2006). These biotic communities influence selection on sunflower traits (Whitney et al. 2006).

\subsection{Study sites and design}
In an observational study in 2009, at three naturally occurring populations of H. a. texanus near to (< 10 m), and four populations far from (~ 2.5 km) crop sunflowers, I marked with metal tags approximately 100 plants/population (Fig 1).  

In experimental studies in 2010, we manipulated the proximity of H. a. texanus to crop sunflowers: Near [array of H. a. texanus � 15 m from the crop] vs. far [array ~2.5 km from any sunflower crop, and near natural habitat].  Plots were replicated at each of five farms in TX (Fig 1.).  The proximity treatment was crossed factorially with a seed origin treatment (seeds from two wild populations collected in 2009; indicated in Fig. 1) to enhance the generality of results.  In 2011, we used the exact same design as 2010, but only used two of the five sites used in 2010 (Marek, Beakley).

\end{document}